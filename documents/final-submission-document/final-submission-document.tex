% !TeX spellcheck = en_GB
%%%%%%%%%%%%%%%%%%%%%%%%%%%%%%%%%%%%%%%%%
% Masters/Doctoral Thesis 
% LaTeX Template
% Version 2.5 (27/8/17)
%
% This template was downloaded from:
% http://www.LaTeXTemplates.com
%
% Version 2.x major modifications by:
% Vel (vel@latextemplates.com)
%
% This template is based on a template by:
% Steve Gunn (http://users.ecs.soton.ac.uk/srg/softwaretools/document/templates/)
% Sunil Patel (http://www.sunilpatel.co.uk/thesis-template/)
%
% Template license:
% CC BY-NC-SA 3.0 (http://creativecommons.org/licenses/by-nc-sa/3.0/)
%
%%%%%%%%%%%%%%%%%%%%%%%%%%%%%%%%%%%%%%%%%
\input{./latex_template}
\loadglsentries{glossar}

\begin{document}

\frontmatter % Use roman page numbering style (i, ii, iii, iv...) for the pre-content pages
\pagestyle{plain} % Default to the plain heading style until the thesis style is called for the body content

%----------------------------------------------------------------------------------------
%	TITLE PAGE
%----------------------------------------------------------------------------------------

\begin{titlepage}
\begin{center}

\vspace*{.06\textheight}
{\scshape\LARGE \univname\par} % University name

{\scshape\large Department of Computer Science\par}\vspace{1.2cm} % University name
\textsc{\Large Bachelor Thesis}\\[0.5cm] % Thesis type

\HRule \\[0.4cm] % Horizontal line
{\huge \bfseries \ttitle\par}\vspace{0.4cm} % Thesis title
\HRule \\[1.5cm] % Horizontal line
 
\begin{minipage}[t]{0.4\textwidth}
\begin{flushleft} \large
\emph{Authors:}\\
\authorname % Author name - remove the \href bracket to remove the link
\end{flushleft}
\end{minipage}
\begin{minipage}[t]{0.4\textwidth}
\begin{flushright} \large
\emph{Advisor:} \\
\supname \\[1cm]
\emph{External Co-Examiner:} \\
\examname \\[1cm]
\emph{Internal Co-Examiner:} \\
Prof.~Dr.~Thomas~Bocek \\[1cm]
\end{flushright}
\end{minipage}\\[3cm]
 
\vfill

{\large Spring Term 2018}\\[4cm] % Date
\includegraphics{resources/logo_hsr} % University/department logo - uncomment to place it
 
\vfill
\end{center}
\end{titlepage}
%----------------------------------------------------------------------------------------
%	License / information PAGE
%------------------------------------------

\vspace*{\fill}

\noindent \textcopyright  Copyright 2018 by Fabian Hauser and Raphael Zimmermann\\

\noindent This documentation is available under the GNU FDL License. \\

\noindent The XMPP-Grid Broker software is licensed under the AGPL-License. This does not apply to third-party libraries.

\pagebreak

%----------------------------------------------------------------------------------------
%	TASK DESCRIPTIONS
%----------------------------------------------------------------------------------------
\includepdf[pages=1,scale=.75,frame, pagecommand=\section*{Task Description}\label{sec:task-description}]{../task-description-signed.pdf}
\addchaptertocentry{Task Description}
\includepdf[pages=2-,scale=.75,frame]{../task-description-signed.pdf}

%----------------------------------------------------------------------------------------
%	ABSTRACT PAGE
%----------------------------------------------------------------------------------------

\begin{abstract}
\addchaptertocentry{\abstractname}% Add the abstract to the table of contents
% Context
The IETF Managed Incident Lightweight Exchange~(MILE) working group proposes the standard ``Using XMPP for Security Information Exchange'' which describes how an XMPP based publish-subscribe mechanism (XMPP-Grid) can be used to exchange security-relevant information between network endpoints.

Today, no implementation of an administration interface (XMPP-Grid Broker) for XMPP-Grids exists, which is production ready and platform independent.\\
\vspace*{1ex}

%Goal
The goal of this thesis is to engineer a specialised and platform independent XMPP-Grid Broker that configures existing XMPP Controllers (XMPP server). The XMPP-Grid Broker should enable administrators to configure XMPP-Grids in a usable and productive way.

The focus of the application should be set on portability, extensibility and the security aspects in a production environment.\\
\vspace*{1ex}

%Result
We propose the architecture of an XMPP-Grid Broker to configure XMPP Controllers and present an implementation based on the Angular5 framework.

The resulting implementation enables administrators to create and configure communication Topics, apprehend the underlying hierarchy and manage permissions. Additionally, persistent items of Topics can be inspected and published.\\
\vspace*{1ex}

% Consequences
The XMPP-Grid Broker implementation incorporates the specified functionality, demonstrating that the architecture is robust. Some supplementary helpers, such as autocomplete or filtering, could not be implemented efficiently due to limitations of the underlying XMPP standards.
\end{abstract}

%----------------------------------------------------------------------------------------
%	MANAGEMENT SUMMARY
%----------------------------------------------------------------------------------------

\chapter{Management Summary}

%The Thesis Management Summary is written here (and usually kept to just this page).
% Das Management Summary richtet sich in der Praxis an die "Chefs des Chefs",
%  d.h. an die Vorgesetzten des Auftraggebers (diese sind in der Regel keine Fachspezialisten).
%  Die Sprache soll knapp, klar und stark untergliedert sein. Zu verwenden ist folgenden Gliederung:
% - Ausgangslage
% - Vorgehen, Technologien
% - Ergebnisse
% - Ausblick (optional)
\section*{Motivation}
%TODO: Can we just use IETF in this context?
The IETF standard draft ``Using XMPP for Security Information Exchange'' describes how devices can exchange security-relevant information within a network, so-called ``XMPP-Grid'', using the widely adopted messaging protocol ``XMPP''.
Such security-relevant information can be used to take protective measures automatically, e.g. to block devices running outdated software.

An administration interface, in the standard draft referred to as XMPP-Grid Broker, is used to configure an XMPP-Grid.
Today, no such XMPP-Grid Broker exists that is production ready and cross-platform.

\section*{Project Goals, Approach}

The goal of this thesis is to engineer an XMPP-Grid Broker that allows administrators to configure XMPP-Grids in a usable and productive way.
The resulting application should only depend on the underlying standards and not on specific implementations to support the further standardisation process.

A comprehensive analysis of the underlying standards is carried out in the first part of this thesis.
Particular attention is given to portability, extensibility and security aspects in a production environment.
By this analysis, a systematic selection of possible architecture options is executed in the form of architectural decisions.

\section*{Results}

The resulting implementation enables administrators to create and configure communication Topics, apprehend the hierarchy of the underlying XMPP-Grid and manage permissions of network participants.
Additionally, persisted messages can be inspected and published on communication Topic.

The XMPP-Grid Broker is implemented as Angular5 web application that connects directly from the web browser to the XMPP-Grid. Secure communication is assured by the use of mutual authentication over TLS.

\section*{Prospects}

The XMPP-Grid Broker implementation incorporates the specified functionality, demonstrating that the architecture is robust.

In the future, the XMPP-Grid Broker might be extended by usability features like autocomplete or filtering. Currently, these features could not be implemented efficiently due to missing support in the underlying XMPP standard, which would have to be extended.

We hope that our work contributes to the further standardisation of XMPP-Grids and that it becomes an established standard used in practical industry applications.

%----------------------------------------------------------------------------------------
%	ACKNOWLEDGEMENTS
%----------------------------------------------------------------------------------------

\begin{acknowledgements}
\addchaptertocentry{\acknowledgementname} % Add the acknowledgements to the table of contents
We would like to thank our advisor, Prof.~Dr.~Andreas~Steffen, for his continuous support and helpful comments.

Tobias~Brunner provided us with valuable feedback on our architecture and introduction section.

Furthermore, we would like to thank Andrea~Jurt~Massey for her feedback regarding writing and language use.
\end{acknowledgements}

%----------------------------------------------------------------------------------------
%	LIST OF CONTENTS PAGES
%----------------------------------------------------------------------------------------

\setcounter{tocdepth}{2}
\tableofcontents % Prints the main table of contents

%----------------------------------------------------------------------------------------
%	DEDICATION
%----------------------------------------------------------------------------------------

%TODO
%\dedicatory{For/Dedicated to/To my\ldots}

%----------------------------------------------------------------------------------------
%	THESIS CONTENT - CHAPTERS
%----------------------------------------------------------------------------------------

\mainmatter % Begin numeric (1,2,3...) page numbering

\pagestyle{thesis} % Return the page headers back to the "thesis" style

% !TeX spellcheck = en_GB
\chapter{Introduction}
\label{sec:introduction}

\section{Motivation}
In this section, we legitimate this work and explain the value and applicability of our proposed solution.

% !TeX spellcheck = en_GB
\chapter{Analysis}

\section{Requirements}
% NFR, priorisierung, Testbarkeit, Accessibility
% Anforderungen abgenommen?

\section{Domain Analysis}


% !TeX spellcheck = en_GB
\chapter{Concept} % Lösungsentwurf 
\epigraph{Perfection (in design) is achieved not when there is nothing more to add,\\but rather when there is nothing more to take away.}{Antoine de Saint-Exupery}

% (Lösungsvarianten und deren Beurteilung, Variantenentscheid, Konzept, Entwurf)
% See arch. decisions

\section{Architecture}
% - Architekturdiagramme inkl. Layering (C4, UML)
% - mit Entscheidungen und begründung
% - wie Qualitätsattribute sichergestellt wurden
% - Beschreibung des Entwurfs (welche Experimente/Tests wurden durchgeführt? welche Lösungsoptionen wurden verworfen?)
% - Entwurf Benutzerschnittstelle

In this chapter, we present the architecture and fundamental architectural decisions of the XMPP grid broker application.
All architectural decisions we took are fully documented in Appendix~\fullref{sec:architectural-decisions}.

We illustrate the concepts and structures using the C4 Model for Software Architecture~\cite{c4-model}.

\subsection{Actors and Context}

The context diagram pictured in Figure~\ref{fig:architecturecontext} shows the surrounding systems and actors that are given for the XMPP grid broker, as described in the \nameref{sec:task-description}.

One or more administrators manage the XMPP grid by adding or removing \glspl{platform} and configuring \glspl{topic}.
To minimize the required work and reduce the error-proneness, they interact with the XMPP grid broker, whose implementation is the primary goal of this thesis.

The XMPP grid broker configures the XMPP grid, which consists of a \gls{controller} and \glspl{platform}.

\begin{figure}[h]
\centering
\includegraphics[width=\linewidth]{resources/architecture_context}
\caption[Architecture Context Diagram]{Architecture diagram showing the context of the XMPP grid broker application.}
\label{fig:architecturecontext}
\end{figure}


\subsection{Architectural Style and Platform}

To implement the XMPP grid broker, we evaluated three possible architecture styles:\hfill\\
An XMPP Server Plugin (e.g. extension for the Openfire XMPP server), an implementation with the Jabber Component Protocol~\cite{xep-0114} or an implementation acting as a regular XMPP client ("bot").

We decided to build an XMPP client/bot, because it is not coupled to a specific XMPP server as the Server Plugin and, in contrast to the XMPP Component, supports a strong authentication mechanism with SASL.

The full decision argument is documented in Appendix~\fullref{sec:architectural-decisions}.

\subsubsection{Platform}

The proposed XMPP client might be implemented in three different ways: as rich client application with a command line or graphical interface as illustrated in Figure~\ref{fig:architecturecontainerrichclient}, or in the form of a web application, illustrated in Figure~\ref{fig:architecturecontainerwebapplication} and \ref{fig:architecturecontainerwebproxy}.

\begin{figure}[h]
\centering
\includegraphics[width=0.7\linewidth]{resources/architecture_container_rich_client}
\caption[Architecture Container Diagram: Rich Client]{Architecture Container Diagram showing a possible rich client architecture.}
\label{fig:architecturecontainerrichclient}
\end{figure}

A web application has the significant advantage to be easily installable, upgradable and has minimal requirements on the user's side (i.e. only requires a web browser to be executed).
Therefore, we decided to implement the XMPP grid broker as web application.

\subsection{Web Application Topology}

To manage the Controller from our interface, we considered the implementation of either directly connecting to the XMPP server over WebSockets~\cite{rfc7395} or HTTP (BOSH~\cite{xep-0124}), or to communicate indirectly with the XMPP server via custom WebAPI Proxy.
These topologies are illustrated in Figure~\ref{fig:architecturecontainerwebapplication} and Figure~\ref{fig:architecturecontainerwebproxy}.

\begin{figure}[h]
\centering
\includegraphics[width=0.7\linewidth]{resources/architecture_container_webapplication}
\caption[Architecture Container Diagram: Web Application]{Architecture Container Diagram showing the web application topology with WebSockets or BOSH.}
\label{fig:architecturecontainerwebapplication}
\end{figure}

\subsubsection{WebAPI Proxy}

A WebAPI Proxy could be realised with a custom browser-to-proxy protocol, as implemented in the XMPP-FTW JavaScript library\footnote{\url{http://docs.xmpp-ftw.org/}}.
However, this approach leads to a high coupling between a concrete library and the web application.

Another approach would be the implementation of a custom WebSocket-to-XMPP Proxy, which allows connecting to XMPP servers that do not fully support WebSockets or BOSH.
If such a proxy is implemented transparently, the client is not aware of the server limitations. Therefore, the client implementation is no different from direct communication with an XMPP server.

\begin{figure}[h]
\centering
\includegraphics[width=\linewidth]{resources/architecture_container_proxy.pdf}
\caption[Architecture Container Diagram: Web Proxy]{Architecture Container Diagram showing the WebAPI Proxy topology.}
\label{fig:architecturecontainerwebproxy}
\end{figure}

\subsubsection{Implemented Web Application Topology}

As elaborated in the according design decision (see Appendix~\ref{sec:architectural-decisions}), we decided for the direct connection via WebSockets, with a fallback to BOSH if the required features are not supported (e.g. SASL EXTERNAL authentication).
This topology simplifies the implementation and deployment of the application in comparison to a WebAPI Proxy.
WebSockets offer stateful TCP-sockets to exchange data with the XMPP server in contrast to BOSH, which uses HTTP long polling to emulate a bidirectional stream and is, therefore, less efficient~\cite{xep-0124}.

Using the XMPP Service Discovery~\cite{xep-0030}, the XMPP server may be queried for supported features, so that only supported functionality is presented to the application user.

\subsection{Authentication and Connection Security}

XMPP uses SASL as authentication mechanism~\cite{rfc6120}.
To authenticate against the XMPP Grid Controller, we decided to use the SASL EXTERNAL~\cite{rfc4422} mechanism whenever possible to authenticate the client.

We decided against the alternative SASL authentication method, SASL SCRAM~\cite{rfc7677}, that is also recommended by in the XMPP grid draft~\cite{ietf-mile-xmpp-grid-05}.
As described in the corresponding architectural decision (Appendix~\ref{sec:architectural-decisions}), the main reason for SASL EXTERNAL is its higher level of security and and its relatively simple scaling capabilities.

SASL EXTERNAL implies that the authentication takes place on a lower layer than the actual XMPP protocol. In our case, this implies authentication over TLS, i.e.~X.509 end-user certificates as specified in RFC6120~\cite{rfc6120}.


\section{Wireframes}

We created wireframes for most screens to visualise the initial set of user stories.
They helped us to find missing requirements, most notably the support of collections.
All wireframes are listed in Appendix~\fullref{sec:wireframes}.

\section{Security Considerations}\label{sec:security-considerations}

Regarding only the XMPP-Grid Broker application, there are three primary attack vectors:

\begin{description}
    \item[Client-side attacks,] e.g. via web browser, web browser extension or malicious software on the client operating system.
    \item[Web server attacks,] e.g. misconfiguration or insufficient hardening.
    \item[XMPP server attacks,] e.g. misconfiguration or insufficient hardening.
\end{description}

Details on all attack vectors are discussed in the following sections.

\subsection{The XMPP Protocol}

An in-depth security analysis of the XMPP protocol is out of the scope of this thesis.
A detailed discussion of security concerns can be found in the XMPP specification~\cite{rfc6120} and most XEPs~\cite{xep-0060}\cite{xep-0248}.
In this section, we highlight the most crucial security concerns relevant to this thesis.

\subsubsection{Transport Security}

XMPP reuses many established and standardised mechanisms to improve the protocol security.
By layering protocols in a strict manner (XMPP with SASL over TLS over TCP), many attack scenarios such as replaying or eavesdropping are minimised.
The protocol also requires clients and servers to validate the certificates of the other party.~\cite{rfc7590}\cite{rfc6120}

\subsubsection{Protocol}

Since XMPP is based on XML, it inherits some of its security implications.
XMPP prohibits some XML features such as comments and external entity references which mitigate common attacks.~\cite{rfc6120}

The protocol itself cannot mitigate attacks where an attacker gains access to account credentials.
To prevent such corruptions best practices such as storing certificates and passwords securely must be followed.

\subsubsection{PubSub Collection Nodes}

The use of PubSub Collection Nodes~\cite{xep-0248} can leak private data if not configured properly.
Administrators must take great care when configuring collection nodes.
The XMPP-Grid Broker should support Administrators to detect such data leaks.

\subsection{Client Security}

Because the web has many potential security concerns, above all a modern web browser is critical for client security.
Legacy browsers can not provide an adequate level of security.~\cite{firefox-update-security}

Most browser support extension mechanisms which have rather significant capabilities.
The usage of untrusted or uncertified browser extensions is strictly discouraged.~\cite{browser-extension-security}

The same applies to the client operating system and all software installed on clients.

\subsubsection{Authentication and Authorisation}

Regarding authentication and authorisation, the XMPP server does most of the heavy lifting such as storing passwords and validating certificates.
On the client side, the browser does most of that work too (i.e. validating certificates).

The responsibility of our client implementation is to establish a secure channel to the XMPP server and warn the user if a problem occurs during this process (e.g. invalid server certificate).

\subsubsection{Angular Framework}

Using the Angular framework impacts client security significantly.
Angular is built with security in mind and is adopted in the industry in security-relevant environments.
Therefore, Angular receives frequent security updates and is well tested.
By using plain JavaScript, it's unlikely to achieve the same level of security in a reasonable timespan.

On their project website, Angular recommends the following three best practices regarding security~\cite{angular-security}.

\begin{itemize}
    \item Keep current with the latest Angular library releases.
    \item Don't modify your copy of Angular.
    \item Avoid Angular APIs marked in the documentation as ''Security Risk''.
\end{itemize}

We can ensure the later two by making them acceptance criteria.
Keeping current with the latest Angular releases is harder, as our work on this project is limited.
To ensure that future updates can easily be applied we deviate as less as possible from the standard angular setup (e.g. by not ejecting the Webpack configuration\footnote{\url{https://github.com/Angular/Angular-cli/wiki/eject}}).

Keeping Angular up-to-date is of paramount importance as potential vulnerabilities (e.g. XSS) can be exploited if not patched.

\subsubsection{Angular Content Security}

Except for \glspl{persisted-item}, no XMPP content is displayed directly but serves as the basis for rendered HTML components.
To protect against malicious payloads, the received XML messages must be validated before their usage.

\Glspl{persisted-item} can contain an arbitrary content and must therefore be escaped before rendering to prevent Cross-Site Scripting (XSS) attacks.

Angular supports these measures by treating all values (except Angular templates) as untrusted by default.
With regards to the Angular templates, we use the offline template compiler, so that no user-generated data can influence them. To fully utilise the security measures provided by Angular, their APIs must be used at all times instead of direct use of the DOM-APIs.~\cite{angular-security}

Using Content-Security-Policy (CSP) provides additional XSS-protection mechanisms \cite{w3c-csp}.
The XMPP-Grid Broker should document an appropriate CSP, that must be supported in a production environment.

\subsection{Server Security}

\subsubsection{Authentication and Authorisation}

The XMPP server implements most of the authentication and authorisation mechanisms used in a XMPP-Grid Broker implementation, such as storing passwords and validating certificates.

If used together with BOSH or WebSockets, it is important that the XMPP server supports most HTTP security features, as listed in Section~\ref{sec:web-server}. Additionally, the origin of WebSockets and BOSH requests must be verified (by either the \texttt{Origin}-Header or CORS support.~\cite{rfc6455}\cite{cross-origin-resource-sharing}

The web server hosting the client application has no active authentication or authorisation responsibility, except to ensure the integrity and authenticity of the application, i.e. by using TLS.

\subsubsection{Web Server}\label{sec:web-server}

To minify security concerns on the server side, we decided to keep the server side static (See \fullref{sec:architectural-decisions}).
This allows operators to use any standard web server (e.g. NGINX, Apache, etc.) to serve the client.
Securing such standard web servers is common knowledge for operators and is out of the scope of this analysis.

In addition to these general best practices, we explicitly recommend the following security measures to maximise the client security:

\begin{itemize}
    \item Enable Content Security Policy (CSP)~\cite{w3c-csp}.
    \item Use secure TLS configurations such as secure Cipher Suites, strictly Honor Cipher Order, HSTS, HPKP and OCSP Stapling\footnote{\url{https://wiki.mozilla.org/Security/Server_Side_TLS}}.
\end{itemize}


\subsubsection{XMPP Server}

The XMPP server security depends on the chosen implementation and the application domain.
Discussing XMPP server security in detail is out of the scope of this thesis.
Operators should adhere to the security recommendations of their XMPP server vendor and follow general security best practices as outlined by the XMPP-Grid draft~\cite{ietf-mile-xmpp-grid-05}.

\section{Security Risk Mitigation}

To mitigate the security risks as discussed in Section~\ref{sec:security-considerations}, we implement the measures as described in the following subsections.

\subsection{Development}

\begin{enumerate}
    \item Conduct Code Reviews for all newly added code using GitHub pull requests and a security checklist (See next section)
    \item Conduct an architectural analysis with an industry expert.
    \item Define explicit security requirements in the form of constraints
    \item Automate build and release processes to minimise the time required to patch.
    \item Stay as close to the default Angular setup to simplify further updates
    \item Avoid additional third-party dependencies whenever possible
\end{enumerate}

\subsection{Client Security Checklist}
\begin{itemize}
    \item The latest Angular-version is used
    \item No customizations are made to the Angular version
    \item No direct access to DOM-APIs
    \item APIs marked in the documentation as ``Security Risk'' are \emph{not} used
    \item No usage of any Methods starting with \texttt{bypassSecurityTrust}.
    \item The client is fully Content Security Policy compliant
    \item The client is fully Same Origin Policy (SOP) / Cross-Origin Resource Sharing (CORS) compliant
    \item Only complete templates offline using the offline template compiler.
    \item User input is always escaped using the mechanisms provided by the framework
    \item XMPP-Messages are validated to contain only the specified result-types
\end{itemize}
% !TeX spellcheck = en_GB
\chapter{Implementation and Testing} % Realisierung und Test
\epigraph{Any fool can write code that a computer can understand. Good programmers write code that humans can understand.}{Martin Fowler}


\section{Development Setup}

Figure~\ref{fig:development-setup} illustrates the development setup in the form of an UML deployment diagram.
A developer connects via his browser to the reverse proxy that serves the XMPP-Grid Broker web application.
The HTTP connection from the client to the server is secured using mutual TLS authentication.
The same reverse proxy also routes the XMPP connections.
The client also authenticates to the proxy using mutual TLS authentication, and the proxy afterwards establishes a TLS connection to the XMPP server using his client certificates.
The reasons for this setup is described in more detail in Section \fullref{encountered-problems}.

\begin{figure}[h]
    \centering
    \includegraphics[width=1\linewidth]{resources/development-setup-uml}
    \caption{UML Deployment Diagram presenting the development setup}
    \label{fig:development-setup}
\end{figure}

As the previously described structure is not trivial, the guiding principle for our development setup was to maximise automation and minimise manual setup and configuration efforts. This principle is the basis for a durable software.
We decided on a docker and docker-compose\footnote{\url{https://www.docker.com/}} based stack that provides a correctly configured Openfire instance, a preconfigured nginx\footnote{\url{https://www.nginx.com/}} instance as well as client and server certificates.
Everyday tasks such as generating new certificates or building and testing the application and documentation were automated as bash scripts.

The efforts invested in this docker setup proved valuable when we began to write integration tests that run in the same environment.

We deliberately decided to run unit tests outside of the docker environment as unit tests are executed more often, and the additional docker-overhead would, therefore, be unnecessarily expensive.
Also, debugging is more straightforward without any indirections.

\section{Encountered Problems}\label{encountered-problems}

\subsection{Limitations of \emph{\fullref{sec:requirement-multiple-administrators}}}\label{sec:limitations-of-requirement-multiple-administrators}

Requirement \ref{sec:requirement-multiple-administrators} states that multiple administrators should be able to access the application.

When authenticating users with SASL EXTERNAL, the client certificate extension field `xmppAddr' is interpreted as user \gls{jid} by the \gls{xmpp} server.

In practice, most \gls{xmpp-grid} \gls{broker} deployments will require an HTTP proxy in front of the \gls{xmpp} server as security measure\footnote{
More information on this can be found in Section~\fullref{sec:implemented-web-application-topology}.}.
Usually, the HTTP proxy can also be used to serve the \gls{broker} application.
Such an HTTP proxy might also accept multiple different client certificates.

If the client connects to the \gls{xmpp} server over secure WebSockets (WSS) in combination with SASL EXTERNAL, the WebSocket URL must already be authenticated, as most browsers do not permit certificate selection on background requests\footnote{\url{https://bugs.chromium.org/p/chromium/issues/detail?id=329884\#c24}}.
This might be achieved by serving the \gls{broker} from the same domain or by using client certificate policies\footnote{\url{https://support.google.com/chrome/a/answer/6080885?hl=en\#manage-certs}}.

As the proxy intercepts the TLS connection, it must verify the client certificate sent by the browser and establish a connection to the \gls{xmpp} server using a client certificate as well.
Therefore, the `xmppAddr' field of the proxy's client ceritifcate is used by the \gls{xmpp} server.
If multiple users should be differentiated on the \gls{xmpp} server, an HTTP proxy might choose different client certificates for connecting to the \gls{xmpp} server based on the web browser's client certificate `xmppAddr'.


\subsection{Limitations of \emph{\fullref{sec:requirement-audit-trail}}}

Actions of administrators should be traceable with an audit trail according to requirement \ref{sec:requirement-audit-trail}.

As outlined in Section~\ref{sec:limitations-of-requirement-multiple-administrators}, practical deployments of \gls{xmpp-grid} \glspl{broker} will mostly use a HTTP proxy.
Additionally, to handling the client authentication, the proxy can be used to keep an audit trail of client requests.
These requests can then be correlated with the query log on the \gls{xmpp} server.

Creating audit trails on the client side does not provide additional safety, as users might prevent trail entries by manipulating the client application.
Therefore, no such mechanism was implemented.


% - Logout (TLS)
% - Initial Topic Consumers and Providers + Initial Topic Consumers and Providers => 2 Step Process!
% - Openfire:
%   - Lost updates with OpenFire
%   - falsche Felder - speziell pubsub#node_type
%   - vgl. https://discourse.igniterealtime.org/t/wrong-field-type-of-pubsub-node-type-and-how-to-update-it/81596
% -> Fehlende Methoden/Funktionalität im Standard
%   - "Liste alle Topics" -> Geht nur hierarchisch
%   - Filtering von Persisted Items

\section{Code Quality}
As our \gls{xmpp-grid} \gls{broker} implementation is intended to be a maintainable, production-ready application rather than a prototype, we placed much emphasis on code quality.
The measures taken can broadly be divided into three categories: technical measures, strategic decisions and processes.

\subsubsection{Technical Measures and Strategic Decisions}
Using Angular and the default Angular CLI was mostly a strategic decision.
Deviating as less as possible from the standard configuration ensures long-term maintainability and relatively straight-forward upgrades to newer Angular versions.
Another benefit of the Angular CLI project setup is that it comes with ``codelyzer''\footnote{\url{http://codelyzer.com/}} (including ``tslint'') for static code analysis and style linting.

Apart from the built-in linting mechanism, we followed Angular's Style Guide~\cite{angular-style-guide}.
Using the JetBrains IDEs (IntelliJ Ultimate and Webstorm)\footnote{\url{https://www.jetbrains.com/}} turned out to be particularly helpful as they give quick feedback for frequent mistakes and even violations of the Angular Style Guide.

We would have prefered to use more tools, especially for code metrics such as Lack of Cohesion of Methods (LCOM), and Afferent/Efferent Coupling.
However, we were not able to find such tools that were actively maintained and work with TypeScript.

\subsubsection{Processes}

On the process side, we tried to work test driven as much as possible.
Doing so turned out to be harder than expected as Angular's component testing infrastructure and the actual calls are sometimes wide apart (see Section~\fullref{sec:testing}).

Another process we heavily relied on were code reviews.
Each change, for the documentation and code, was reviewed using GitHub pull-requests\footnote{\url{https://www.github.com/}}.
In most cases, minor changes were detected and addressed during these reviews.
Continuous integration with TravisCI\footnote{\url{https://travis-ci.com/}} ensured that these changes never contained compilation errors or failing tests.

We also regularly discussed architectural and structural questions in our retrospectives and standup meetings.

In general, writing clean, modular and testable code has been our main priority.

\section{Testing}\label{sec:testing}

Good tests are inevitable for long-lived software projects.
They help developers to ensure that everything (still) works as expected after a change.
For the \gls{xmpp-grid} \gls{broker}, we focused on unit and end-to-end tests.
Following the principles of the test pyramid \cite{Cohn:2009:SAS:1667109}, we wrote many cheap and fast unit tests verifying the fundamental behaviour and fewer expensive and complex end to end tests.

\subsubsection{Unit Tests}

Testing the Angular Services was rather straightforward just by using jasmine and its mocking functionality.
We deliberately abstained from using Angulars testing framework for these services to keep them simple and comprehensible.
However, most of the services mainly send and receive XMPP-commands for which integration or end to end tests are required.

Writing tests for Angular Components, which require actual rendering a web browser, was not always essential.
To have fine-grained control and to be able to conduct tests, Angular provides a rather complex set of testing tools.
Because of this indirection, tests are conceptually not identical with the actual Angular application, making test driven development harder if not impossible.

\subsubsection{End to End Tests}

The end-to-end tests were written using Protractor, Angulars official end-to-end testing framework.
Protractor starts the development setup and verifies the application using a remote-controlled browser.

End-to-end tests are usually more challenging to write than unit tests, as different types of race conditions and varying delays to backend applications can occur.
Protractor usually resolves these issue due to its use of Zone.js, a library that creates ``execution context[s] that persists across async tasks'' called Zones.
To create Zones, Zone.js intercepts most web browser events, like HTTP requests.~\cite{zone-js-readme}

Because Zone.js is aware of all open HTTP requests, protractor can wait until a request and document change has completed before continuing with test execution.

However, due to our use of BOSH in the end-to-end tests\footnote{See Section~\fullref{sec:implemented-web-application-topology}}, we could not benefit from the Zone.js change detection.
BOSH uses HTTP long polling to communicate with the XMPP server, which leads to a Zone that always has open requests~\cite{xep-0124}.

Therefore, we had to manually implement waiting conditions, which proved to be difficult due to unstable behaviour of the application stack in different environments.

In our case, writing tests paid off quickly as they promptly caught many potential bugs introduced by small changes and refactorings.

\section{Documentation}

Install instructions, as well as security best practices, are directly documented in the git source code repository using a plain text file format called AsciiDoc.
Interested parties can browse the documentation directly on GitHub, which is not uncommon in the open source community.

A compact getting started guide for developers is also available in the source code repository.
The source code has jsdoc\footnote{\url{http://usejsdoc.org/}} based documentation optimised for compodoc, a ``documentation tool for your Angular applications''\footnote{\url{https://compodoc.app/}}.

As already discussed in Section \fullref{sec:architecture}, all architectural decisions were documented systematically.
These decisions enable new developers and interested parties to comprehend why certain decisions were made.
With the idea of making project documentation durable, all decisions were written in the same plaintext format as the other project documentation.

% !TeX spellcheck = en_GB
\chapter{Discussion and Conclusion}
\epigraph{People do not like to think. If one thinks, one must reach conclusions. Conclusions are not always pleasant.}{\textit{Helen Keller}}
% Alternative:  An expert is a man who has stopped thinking because 'he knows.' (1957)  - Frank Lloyd Wright
% Alternative: Life is the art of drawing sufficient conclusions from insufficient premises. Samuel Butler 
\section{Achieved Result}
% - requirements
% - client
% - tests
% - architecture
% - performance in terms of speed, scalability, concurrency, usability etc
% Ausgewählte Implementierungsdetails/Metriken diskutieren (Bsp. Algorithmen, Datenstrukturen, Libraries, Architectural Hot Spots) 

\section{Discussion}
% Ergebnisdiskussion:
% - Stärken und Schwächen der Konzepte
% - Verbesserungen für die  Zielgruppe im Kontext)

\section{Lessons Learned}
% eg. We tried to systematically document arch. decisions, how did that turned out?

\section{Future work}

\section{Conclusion}


\subsubsection*{} %This is needed, so that not the next page is referenced.
\label{lastpage} %TODO: This label should be positioned below above the last paragraph

\cleardoublepage
%----------------------------------------------------------------------------------------
%	BIBLIOGRAPHY
%----------------------------------------------------------------------------------------
\backmatter
\pagenumbering{Roman}

\bibliographystyle{abbrv}
\bibliography{references}
\addcontentsline{toc}{chapter}{Bibliography}


%----------------------------------------------------------------------------------------
%	LIST OF FIGURES/TABLES PAGES
%----------------------------------------------------------------------------------------

\listoffigures % Prints the list of figures

\listoftables % Prints the list of tables

%----------------------------------------------------------------------------------------
%	GLOSSARY
%----------------------------------------------------------------------------------------

\glsaddall
\printglossary


%----------------------------------------------------------------------------------------
%	THESIS CONTENT - APPENDICES
%----------------------------------------------------------------------------------------


\appendix % Cue to tell LaTeX that the following "chapters" are Appendices
\chapter{Appendices}
\setcounter{secnumdepth}{3}
\renewcommand{\thechapter}{A}
\section{Project Plan}\label{sec:project-plan}
\includepdf[pages=-,scale=.9,frame]{../project-plan/project-plan.pdf}
\section{Development Guide}\label{sec:development-guide}
\includepdf[pages=-,scale=.9,frame]{../development-guide.pdf}
\section{Architectural Decisions}\label{sec:architectural-decisions}
\includepdf[pages=-,scale=.9,frame]{../architectural-decisions/architectural-decisions.pdf}
\section{Time Accounting}\label{sec:time-accounting}
\includepdf[pages=-,scale=.9,frame]{../time-accounting.pdf}
\section{Meeting Minutes}\label{sec:meeting-minutes}
\includepdf[pages=-,scale=.9,frame]{../meeting-minutes/meeting-minutes.pdf}

% !TeX spellcheck = en_GB

\section{Requirements}\label{sec:requirements}

The following sections describe the primary requirements in the form of user stories~\cite{agile-alliance-user-stories}.
Figure~\ref{fig:requirements-overview} shows an overview of the primary use stories.

\begin{figure}[h]
    \centering
    \includegraphics[width=1\linewidth]{resources/requirements_overview}
    \caption{UML Use Case Diagram presenting an overview of the primary user stories.}
    \label{fig:requirements-overview}
\end{figure}

\subsection{Authentication}
\subsubsection{Login}

As an Administrator,\\
I want to log in\\
- preferably using an existing client TLS certificate - \\
so that only I can inspect and manage Topics.\\

\subsubsection{Secure \gls{xmpp} Authentication}

As an Administrator concerned with security requirements,\\
I want to use either SASL EXTERNAL or SASL SCRAM mechanism for authentication -

\begin{itemize}
    \item preferably the SCRAM-SHA-256-PLUS variant and
    \item preferably using mutual certificate-based authentication including revocation status checking
\end{itemize}

\noindent - so that the Controller is fully compatible with the \gls{xmpp-grid} standard~\cite{ietf-mile-xmpp-grid-05}.

\noindent To achieve this goal, I am willing to accept:
\begin{itemize}
    \item More costly and less user friendly authentication
    \item limited compatibility of supported \gls{xmpp} servers
\end{itemize}

\subsubsection{Secure \gls{xmpp} Connection}

As an Administrator concerned with security requirements,\\
I want to use minimally TLS 1.2 [RFC5246] to communicate with the \gls{xmpp} server at all times\\
to achieve maximal security and compatibility with the \gls{xmpp-grid} standard~\cite{ietf-mile-xmpp-grid-05}.

\subsubsection{Secure Connection}

As an Administrator concerned with security requirements,\\
I want to use minimally TLS 1.2 [RFC5246] to communicate with the Broker\\
to achieve maximal security.

\subsubsection{Multiple Administrations}

As an Administrator,\\
I want to grant access to administrators \\
so that they can also manage the application.

\subsubsection{Audit Trail}

As an Administrator concerned with security requirements,\\
I want to be able to access an audit log\\
- preferably using existing \gls{xmpp} mechanisms - \\
so that I can reconstruct what other Administrations did on the Controller.

\subsubsection{Logout}

As an Administrator,\\
I want to log out\\
so that I can terminate a session.

\subsection{List Topics and Collections}

\subsubsection{List All Topics}
As an Administrator,\\
I want to see a list of all Topics of the associated Controller\\
so that I can quickly assimilate which Topics exist.

\subsubsection{List All Top-Level-Collections}
As an Administrator,\\
I want to see a list of all Top-Level-Collections of the associated Controller\\
so that I can quickly assimilate which Collections exist.

\subsubsection{List All Parent-Collections of a Topic}
As an Administrator,\\
I want to see a list of all transitive parent Collections that contain a given Topic\\
so that I can quickly assimilate in which Collections items are published.

\subsubsection{List All Subtopics and Subcollection of a Collection}
As an Administrator,\\
I want to see a list of all Collections and Topics that a given Collection contains\\
so that I can quickly assimilate the collection hierarchy.

\subsubsection{List Available Topics With Limited Access (optional)}

As an Administrator,\\
I want to see a list of all Topics of the associated Controller to which I have limited access to,\\
to simplify troubleshooting and locate errors.

\subsubsection{List Available Collections With Limited Access (optional)}

As an Administrator,\\
I want to see a list of all Collections of the associated Controller to which I have limited access to,\\
to simplify troubleshooting and locate errors.

\subsubsection{Topic and Collection Paging}
As an Administrator,\\
I want to be able to page through any set of Collection/Topic with more than 10 Items \\
so that I can work with more than 1000 Collections and Topics more effectively.

\subsubsection{Topic and Collection Name Filter}
As an Administrator,\\
I want to be able to quickly filter any set of Collections/Topics with more than 10 Items \\
so that I can work with more than 1000 Collections and Topics more effectively.

\subsection{Create a New Topic}

As an Administrator,\\
I want to create a new Topic on the associated Controller\\
so that I am not tied to a fixed set of Topics.

\subsection{Create a New Collection}

As an Administrator,\\
I want to create a new Collection on the associated Controller\\
so that I can flexibly patch Topics together.

\subsubsection{Override Default Topic Configuration}

As an Administrator in the process of creating a new Topic,\\
I want to override the default configuration (e.g. the affiliations) \\
so that I can restrict access and provide reasonable defaults.

\subsubsection{Override Default Collection Configuration}

As an Administrator in the process of creating a new Collection,\\
I want to override the default configuration (e.g. the affiliations) \\
so that I can restrict access and provide reasonable defaults.

\subsubsection{Initial Topic Consumers and Providers}

As an Administrator in the process of creating a new Topic,\\
I want to specify an initial set of Consumers and Providers \\
so that I can restrict access to that Topic and provide reasonable defaults.

\subsubsection{Initial Collection Consumers}

As an Administrator in the process of creating a new Collection,\\
I want to specify an initial set of Consumers \\
so that I can restrict access to that Collection and provide reasonable defaults.

\subsection{Delete an Existing Topic}

As an Administrator,\\
I want to delete an existing Topic on the associated Controller\\
so that I can get rid of obsolete Topics.

\subsection{Delete an Existing Collection}

As an Administrator,\\
I want to delete an existing Collection on the associated Controller\\
so that I can get rid of obsolete Collections.


\subsubsection{Fault Prevention On Topic-Delete}

As an Administrator in the process of deleting a Topic, \\
I want a mechanism to prevent me from deleting the wrong Topic on the associated Controller\\
(e.g. require me to enter the name of the Topic manually).

\subsubsection{Fault Prevention On Collection-Delete}

As an Administrator in the process of deleting a Collection, \\
I want a mechanism to prevent me from deleting the wrong Collection on the associated Controller\\
(e.g. require me to enter the name of the Collection manually).


\subsection{Manage Topic/Collection Subscriptions}

\subsubsection{List Consumers}

As an Administrator, \\
I want to list all Consumers (including their JIDs) of a given Topic/Collection on the associated Controller, \\
so that I can verify that specific Consumers are subscribed, and others are not.


\subsubsection{Inspect Detailed Subscription Configuration}

As an Administrator, \\
I want to inspect the detailed Topic/Collection subscription configuration of a given Consumer, \\
so that I can reproduce and reason about the receipt of data on that Consumer
and find potential misconfiguration.

\subsubsection{Partially Modify Subscription Configuration}

As an Administrator, \\
I want to modify parts of the Topic/Collection subscription configuration of a given Consumer, \\
so that I can fix misconfiguration.

\subsubsection{Unsubscribe Consumer}

As an Administrator, \\
I want to manually unsubscribe a specific Consumer from a particular Topic/Collection on the associated Controller, \\
so that I can remove obsolete or undesired subscriptions.

\subsubsection{Subscribe Consumer}

As an Administrator, \\
I want to manually subscribe a specific Consumer on a particular Topic/Collection on the associated Controller, \\
so that I can faster setup and manage Consumers.

\subsection{Manage Topic Affiliations}
\subsubsection{Inspect Affiliations}

As an Administrator,\\
I want to list all Affiliations (JID and "Role") for a particular Topic/Collection on the associated Controller \\
so that I can find potential misconfiguration.

\subsubsection{Modify Affiliations}

As an Administrator,\\
I want to modify the Affiliation ("Role") of a given JID for a particular Topic/Collection on the associated Controller \\
so that I can fix potential misconfiguration.

\subsubsection{Fault Prevention When Modifying My Affiliation}

As an Administrator in the process of modifying my Affiliation for a particular Topic/Collection on the associated Controller,\\
I want a mechanism to prevent me from accidentally downgrading my rights.

\subsubsection{Meaningful Error For Topics/Collection With Limited Access}

As an Administrator,\\
I want to receive a meaningful error message when inspecting a Topic/Collection to which I have limited access \\
so that I can quickly comprehend why the configuration options are limited.

\subsection{Manage Persisted Items of a Topic}
\subsubsection{Inspect Persisted Items}

As an Administrator,\\
I want to list all Persisted Items for a particular Topic on the associated Controller \\
so that I can get an overview and check for misconfiguration.

\subsubsection{Filter Persisted Items}

As an Administrator,\\
I want to be able to filter all persisted Items of a specific Topic by \\
\begin{itemize}
    \item the timestamp of its publication
    \item the publishers JID
\end{itemize}
so that I can work with more than 10000 persisted items more effectively.

\subsubsection{Paged Persisted Items}
As an Administrator working with filtered persisted items,\\
I want to be able to page through the resulting items\\
- given that this feature is supported by the associated Controller -\\
so that I can work with more than 10000 persisted items more effectively.

\subsubsection{Delete a Persisted Item From a Topic}

As an Administrator,\\
I want to delete a particular persisted item from a specific Topic\\
- given that this feature is supported by the associated Controller -\\
so that I can clean up test items and remove obsolete or corrupted items.

\subsubsection{Purge All Persisted Items From a Topic}

As an Administrator,\\
I want to purge persisted items from a specific Topic\\
- given that this feature is supported by the associated Controller -\\
so that I can clean up test items and remove obsolete or corrupted items.

\subsubsection{Delete Set of Persisted Item From a Topic (optional)}

As an Administrator,\\
I want to delete a set of persisted item that match a given criteria from a specific Topic\\
- given that this feature is supported by the associated Controller -\\
so that I can clean up test items and remove obsolete or corrupted items.

\subsection{Manage Subscription Requests (optional)}

\subsubsection{List Subscription Request}
As an Administrator,\\
I want to list pending subscription requests for a given Topic\\
- given that this feature is supported by the associated Controller -\\
so that I can quickly assimilate pending requests.

\subsubsection{Accept Subscription Request}

As an Administrator,\\
I want to accept a pending subscription request for a given Topic\\
- given that this feature is supported by the associated Controller -\\
to enable more dynamic access models than just maintaining a black- or whitelist.

\subsubsection{Reject Subscription Request}

As an Administrator,\\
I want to reject a pending subscription request for a given Topic\\
- given that this feature is supported by the associated Controller -\\
so that I can deny user access in accordance with the \gls{xmpp} standards.

\subsection{Validate Controller Configuration (optional)}

\subsubsection{Validate Supported XEPs Configurations}
As an Administrator,\\
I want to validate that a minimum set of XEPs are supported by the associated Controller\\
so that I can quickly identify incompatibilities.

\subsubsection{Validate Optional XEP Implementations}
As an Administrator,\\
I want to validate that the required features that are marked as optional or recommended in the XEPs are implemented by the associated Controller\\
so that I can quickly identify incompatibilities.


% !TeX spellcheck = en_GB

\section{Wireframes}\label{sec:wireframes}

\begin{figure}[h]
    \centering
    \includegraphics[width=1\linewidth]{resources/wireframe_1}
    \caption{Login-Screen Wireframe}
\end{figure}

\begin{figure}[h]
    \centering
    \includegraphics[width=1\linewidth]{resources/wireframe_2}
    \caption{Controller Overview Wireframe}
\end{figure}

\begin{figure}[h]
    \centering
    \includegraphics[width=1\linewidth]{resources/wireframe_3}
    \caption{New Topic Wireframe}
\end{figure}

\begin{figure}[h]
    \centering
    \includegraphics[width=1\linewidth]{resources/wireframe_4}
    \caption{Topic Subscriptions Wireframe}
\end{figure}

\begin{figure}[h]
    \centering
    \includegraphics[width=1\linewidth]{resources/wireframe_5}
    \caption{Topic Affiliations Wireframe}
\end{figure}

\begin{figure}[h]
    \centering
    \includegraphics[width=1\linewidth]{resources/wireframe_6}
    \caption{Topic Configuration Wireframe}
\end{figure}

\begin{figure}[h]
    \centering
    \includegraphics[width=1\linewidth]{resources/wireframe_7}
    \caption{Persisted Items Wireframe}
\end{figure}
% !TeX spellcheck = en_GB
\section{Comparison of XMPP Server and Libraries}\label{sec:comparison-of-xmpp-server-and-libraries}


\subsection{Server}
%TODO: - considered only mayor open source servers with good support/maintenance

\subsubsection{Openfire}
\begin{tabu}{l X}
    Programming Language
    & Java \\

    Plugin Architecture
    & Java JAR\footnote{\url{http://download.igniterealtime.org/openfire/docs/latest/documentation/plugin-dev-guide.html}} \\

    Supported XEPs\footnote{\url{http://download.igniterealtime.org/openfire/docs/latest/documentation/protocol-support.html}}
    & \begin{tabu}{@{}l X}
    XEP-0004 & Full\\
    XEP-0114 & Full\\
    XEP-0030 & Full\\
    XEP-0059 & Full\\
    XEP-0060 & Full\\
    XEP-0133 & Partial\footnote{\url{https://issues.igniterealtime.org/browse/OF-284}}\\
    XEP-0248 & Partial\footnote{\url{https://igniterealtime.jiveon.com/thread/38929}} \\
    RFC-7395 & Full\\
    \end{tabu} \\
\end{tabu}

\subsubsection{Prosody}
\begin{tabu}{l X}
    Programming Language
    & Lua \\

    Plugin Architecture
    & Luascript\footnote{\url{https://prosody.im/doc/developers/modules}} \\

    Supported XEPs\footnote{\url{https://prosody.im/doc/modules} and  \url{https://prosody.im/doc/xeplist}}
    & \begin{tabu}{@{}l X}
    XEP-0004 & Full\\
    XEP-0114 & Full\\
    XEP-0030 & Full\\
    XEP-0059 & \emph{Not supported}\\
    XEP-0060 & Full\\
    XEP-0133 & Full\\
    XEP-0248 & \emph{Not supported}\\
    RFC-7395 & Full\\
    \end{tabu} \\
\end{tabu}

\subsubsection{Ejabberd}
\begin{tabu}{l X}
    Programming Language
    & Erlang \\

    Plugin Architecture
    & Erlang/Elixir\footnote{\url{https://docs.ejabberd.im/developer/extending-ejabberd/modules/}} \\

    Supported XEPs\footnote{\url{http://www.ejabberd.im/protocols}}
    & \begin{tabu}{@{}l X}
    XEP-0004 & Full\\
    XEP-0030 & Full\\
    XEP-0059 & Full\\
    XEP-0060 & Full\\
    XEP-0114 & Full\\
    XEP-0133 & Full\\
    XEP-0248 & Full\\
    RFC-7395 & Probably\footnote{\url{https://docs.ejabberd.im/xmpp}}\\
    \end{tabu} \\
\end{tabu}

\subsection{Libraries}

\begin{sidewaystable}
    \centering
    \caption{Comparison of XMPP Client Libraries}
    \label{tbl:language-comparison-ds}
    \begin{tabu}{X|X X X X}
        \hline
        \diagbox[width=9em,height=2.5em]{Name}{Features}
        & Prog. Language
        & Plugin Architecture
        & Req. XEPs
        & Authentication options
        \\ \hline

        SleekXMPP
        & Python 2
        & Yes\footnote{\url{http://sleekxmpp.com/create_plugin.html}}
        & XEP-0060 client only\footnote{\url{http://sleekxmpp.com/xeps.html}}
        & ?
        \\

        SliXMPP
        & Python 3
        & Yes\footnote{\url{https://github.com/poezio/slixmpp/blob/master/docs/create_plugin.rst}}
        & XEP-0060 client only\footnote{\url{https://github.com/poezio/slixmpp/blob/master/docs/xeps.rst}}
        & ?
        \\

        aioxmpp
        & Python 3.4
        & Yes\footnote{\url{https://docs.zombofant.net/aioxmpp/devel/api/public/index.html\#apis-mainly-relevant-for-extension-developers}}
        & Yes\footnote{\url{https://docs.zombofant.net/aioxmpp/devel/\#from-xmpp-extension-proposals-xeps}}
        & ?
        \\

        Smack
        & Java
        & Yes\footnote{\url{https://github.com/igniterealtime/Smack/tree/master/documentation}}
        & Yes\footnote{\url{https://download.igniterealtime.org/smack/docs/latest/documentation/extensions/index.html}}
        & ?
        \\

        Babbler
        & Java
        & Yes\footnote{\url{https://sco0ter.bitbucket.io/babbler/customextensions.html}}
        & Yes\footnote{\url{https://sco0ter.bitbucket.io/babbler/xeps.html}}
        & ?
        \\

        XMPP-FTW
        & Javascript (Browser)
        & Yes\footnote{\url{http://docs.xmpp-ftw.org/}}
        & Yes\footnote{\url{http://docs.xmpp-ftw.org/manual/}}
        & ?
        \\
    \end{tabu}
\end{sidewaystable}


%----------------------------------------------------------------------------------------
%	DECLARATION PAGE
%----------------------------------------------------------------------------------------

\begin{declaration}
\addchaptertocentry{\authorshipname} % Add the declaration to the table of contents
\noindent We, \authorname, declare that this thesis and the work presented in it are our own, original work.  All the sources we consulted and cited are clearly attributed. We have acknowledged all main sources of help. \\

\noindent Fabian Hauser\\[2em]
\rule[0.5em]{25em}{0.5pt}\\ % This prints a line for the signature
\noindent Raphael Zimmermann\\[2em]
\rule[0.5em]{25em}{0.5pt}\\ % This prints a line for the signature 
\noindent Rapperswil, \today
\end{declaration}

\cleardoublepage


%----------------------------------------------------------------------------------------

\end{document}  
