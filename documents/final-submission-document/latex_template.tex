
%----------------------------------------------------------------------------------------
%	PACKAGES AND OTHER DOCUMENT CONFIGURATIONS
%----------------------------------------------------------------------------------------

\documentclass[
11pt, % The default document font size, options: 10pt, 11pt, 12pt
oneside, %TODO: % Two side (alternating margins) for binding by default, uncomment to switch to one side
english, % ngerman for German
singlespacing, % Single line spacing, alternatives: onehalfspacing or doublespacing
%draft, % Uncomment to enable draft mode (no pictures, no links, overfull hboxes indicated)
%nolistspacing, % If the document is onehalfspacing or doublespacing, uncomment this to set spacing in lists to single
liststotoc, % Uncomment to add the list of figures/tables/etc to the table of contents
toctotoc, % Uncomment to add the main table of contents to the table of contents
%parskip, % Uncomment to add space between paragraphs
%nohyperref, % Uncomment to not load the hyperref package
headsepline, % Uncomment to get a line under the header
%chapterinoneline, % Uncomment to place the chapter title next to the number on one line
consistentlayout, % Uncomment to change the layout of the declaration, abstract and acknowledgements pages to match the default layout
]{./latex_template} % The class file specifying the document structure

%\usepackage[utf8]{inputenc} % Required for inputting international characters
\usepackage[T1]{fontenc} % Output font encoding for international characters
\usepackage{mathpazo} % Use the Palatino font by default
\usepackage{tabu}
\usepackage{diagbox}
\usepackage{rotating}
\usepackage{makecell}
\usepackage{float}
\usepackage{url}
\usepackage{xstring}
\usepackage{pdfpages}
\usepackage[normalem]{ulem}
\usepackage[autostyle=true]{csquotes} % Required to generate language-dependent quotes in the bibliography
\usepackage{letltxmacro}

\newcommand{\epigraph}[2]{\vspace{-1em}\begin{flushright}%
\rightskip=1.8cm\textit{``#1''} \\%
\vspace{.2em}%
\rightskip=.8cm--- \textup{#2}%
\end{flushright}%
\vspace{2em}}

\newcommand{\beautyquote}[2]{\begin{flushright}%
\rightskip=1.8cm\textit{``#1''} \\%
\vspace{.2em}%
\rightskip=.8cm--- \textup{#2}%
\end{flushright}}

% \addPdfAppendix[title]{label}{pdf}
\newcommand*{\addPdfAppendix}[4][multipage]{
\includepdf[pages=1,scale=.75,frame, pagecommand=\section{#2}\label{#3}]{#4}
\IfStrEq{#1}{multipage}{
\includepdf[pages=2-,scale=.8,frame]{#4}
}
}

%----------------------------------------------------------------------------------------
%	MARGIN SETTINGS
%----------------------------------------------------------------------------------------

\geometry{
	paper=a4paper, % Change to letterpaper for US letter
	inner=2.5cm, % Inner margin
	outer=3.8cm, % Outer margin
	bindingoffset=.5cm, % Binding offset
	top=1.5cm, % Top margin
	bottom=1.5cm, % Bottom margin
	%showframe, % Uncomment to show how the type block is set on the page
}
\global\tabulinesep=1.2mm

%----------------------------------------------------------------------------------------
%	Glossary SETTINGS
%----------------------------------------------------------------------------------------
\usepackage{hyperref}

\usepackage[toc,nopostdot, nonumberlist]{glossaries}%acronym
\setglossarystyle{altlist}
\usepackage{xparse}
\DeclareDocumentCommand{\newdualentry}{ O{} O{} m m m m } {
	\newglossaryentry{gls-#3}{
		name={#4 : #5},
		text={#5\glsadd{#3}},
		description={#6},
		#1
	}
	\makeglossaries
	\newacronym[see={[Siehe:]{gls-#3}},#2]{#3}{#4}{#5\glsadd{gls-#3}}
}
\renewcommand{\glstextformat}[1]{\slshape{#1}}

\makeglossaries

\DeclareTextFontCommand{\emph}{\itshape}

%----------------------------------------------------------------------------------------
%	THESIS INFORMATION
%----------------------------------------------------------------------------------------

\thesistitle{XMPP-Grid Broker} %is used in the title and abstract, print it elsewhere with \ttitle
\supervisor{Prof.~Dr.~Andreas~Steffen} %is used in the title page, print it elsewhere with \supname
\examiner{Dr.~Ralf~Hauser, PrivaSphere~AG} %print it elsewhere with \examname
\author{Fabian~\textsc{Hauser} and Raphael~\textsc{Zimmermann}} %is used in the title page and abstract, print it elsewhere with \authorname

\keywords{XMPP Grid Broker} % is not currently used anywhere in the template, print it elsewhere with \keywordnames
\university{\href{https://www.hsr.ch}{University of Applied Sciences Rapperswil}} %is used in the title page and abstract, print it elsewhere with \univname
\department{Department of Computer Science} %is used in the title page and abstract, print it elsewhere with \deptname

\AtBeginDocument{
\hypersetup{pdftitle=\ttitle} % Set the PDF's title to your title
\hypersetup{pdfauthor=\authorname} % Set the PDF's author to your name
\hypersetup{pdfkeywords=\keywordnames} % Set the PDF's keywords to your keywords
% Redefine References to be cursive
\LetLtxMacro{\origref}{\ref}
\renewcommand{\ref}[1]{\textsl{\origref{#1}}}
\LetLtxMacro{\orignameref}{\nameref}
\renewcommand{\nameref}[1]{\textsl{\orignameref{#1}}}
\newcommand*{\fullref}[1]{\textsl{\hyperref[{#1}]{\origref*{#1} \orignameref*{#1}}}} % One single link
}
