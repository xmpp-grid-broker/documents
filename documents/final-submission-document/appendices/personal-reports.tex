% !TeX spellcheck = en_GB
\section{Personal Reports}\label{sec:personal-reports}

\subsection{Raphael Zimmermann}

This project was an exciting journey for me because I wasn't familiar with the subject matter and the XMPP protocol at all.

I was surprised that many of the XEPs we used were still in a draft state even though they were around for over ten years and implemented in most servers.
It was even more surprising to me that these XEPs were modified significantly, contradicting my wishful thinking to rely on the specification entirely.
I also learned, that having too many optional features in a standard makes working with it tedious, especially if this functionality seems pivotal.

Because I had no prior experience with Angular, it took much work to understand all relevant underlying concepts.
In hindsight, I must admit that I underestimated the complexity and familiarisation period.
As a result, I was shifting my focus unintentionally from other practices that I usually focus on, such as clean layering.
Nonetheless, I am glad to get familiar with Angular and I mostly enjoyed working with it.

Fabian and I are a well-practised team and as in our study project, working together was a pleasure.

\subsection{Fabian Hauser}

\beautyquote{The important thing is not to stop questioning.}{Albert Einstein}
%
I find this quote very fitting for our thesis - not only was it our task to decide on options,
but also keep questioning them to find the best possible solution.

We used architectural design decisions to find and question possible solutions.
This technique supported our problem solving process greatly in my opinion.
The most surprising turn resulting from an architectural decision during the project was our decision to write a client only application,
which I wouldn't have expected from the task description.

During the project, I often had the feeling that we didn't advance as fast as I had hoped for.
I think the main reason for this feeling is the time it took to get familiar with the complex XMPP standards.

Working together with Raphael was a very pleasant experience.
Although we often worked remote from home, we had great discussions and conversations.
Nevertheless, I think that it was helpful that we took the time to meet at least one time a week, which improved our communication and team spirit.