% !TeX spellcheck = en_GB
\chapter{Implementation and Testing} % Realisierung und Test
\epigraph{Any fool can write code that a computer can understand. Good programmers write code that humans can understand.}{Martin Fowler}

\section{Problems}

\subsection{Limitations of \emph{\fullref{sec:requirement-multiple-administrators}}}\label{sec:limitations-of-requirement-multiple-administrators}

Requirement \ref{sec:requirement-multiple-administrators} states that multiple administrators should be able to access the application.

Authenticating multiple users with SASL EXTERNAL is possible with the client certificate extension field `xmppAddr'.
The value of this field is then interpreted as user \gls{jid} by the \gls{xmpp} server.

In practice, most \gls{xmpp-grid} \gls{broker} deployments will require a HTTP proxy in front of the \gls{xmpp} server as security measure and to host the \gls{broker} application\footnote{
More information on this can be found in Section~\fullref{sec:implemented-web-application-topology} and Section Webapplication Communication Topology in Appendix~\fullref{sec:architectural-decisions}.}.
Such a HTTP proxy might also accept multiple different client certificates.

However, as the proxy authenticates itself against the \gls{xmpp} server with a client certificate as well, the `xmppAddr' value of the proxy certificate is used by the \gls{xmpp} server.
If multiple users should be differentiated on the \gls{xmpp} server, a HTTP proxy might choose different client certificates for connecting to the \gls{xmpp} server based on the browser's client certificate `xmppAddr'.


\subsection{Limitations of \emph{\fullref{sec:requirement-audit-trail}}}

Actions of administrators should be traceable with an audit trail according to requirement \ref{sec:requirement-audit-trail}.

As outlined in Section~\ref{sec:limitations-of-requirement-multiple-administrators}, practical deployments of \gls{xmpp-grid} \glspl{broker} will mostly use a HTTP proxy.
Additionally to handling the client authentication, the proxy might be used to keep an audit trail of client requests which can be correlated with the query log on the \gls{xmpp} server.

Creating audit trails on the client side does not provide additional safety, as users might prevent trail entries by manipulating the client application. Therefore, no such mechanism was implemented.


% - Logout (TLS)
% - Initial Topic Consumers and Providers + Initial Topic Consumers and Providers => 2 Step Process!
% - Openfire:
%   - Lost updates with OpenFire
%   - falsche Felder - speziell pubsub#node_type
%   - vgl. https://discourse.igniterealtime.org/t/wrong-field-type-of-pubsub-node-type-and-how-to-update-it/81596
% -> Fehlende Methoden/Funktionalität im Standard
%   - "Liste alle Topics" -> Geht nur hierarchisch
%   - Filtering von Persisted Items

\section{Code Quality}
% - Qualität des Codes
    % - Metriken
    % - Coding Guidelines
    % - Coding Reviews Richtlinien
    % - JavaDoc etc
    % - Patterns
    % - Code stimmt mit Arch. überein

% Typescript mit liter (tslint)
% Best-Practices von Angular befolgt dank Intellij Ultimate (`@angular/language-service`) und https://github.com/mgechev/codelyzer (ink. tslint)
% Modular und Testbar
% Doku mit Compodoc

\section{Complexity}
% - Komplexität und Umfang
    % - Metriken
\section{Testing}
% - Test
    % - Sinnvoll
    % - Abdeckung der Anforderungen
    % - Coverage
    % - Protokolle

% Viele Angular Component Tests -> Leider grosser gap zwischen test env und realität was testing etwas mühsam macht
% Integration/E2E-Tests mit Protractor
% - mühsam wegen BOSH da immer "polling"

\section{Documentation}
% - Installation
% - Benutzerdokumentation
% - Zielerreichung