% !TeX spellcheck = en_GB
\chapter{Implementation and Testing} % Realisierung und Test
\epigraph{Any fool can write code that a computer can understand. Good programmers write code that humans can understand.}{Martin Fowler}


\section{Development Setup}
% - e2e tests
% - angular cli
% - build script / ci
% - docker-compose, CA automated with scripts etc. SE stuff

\section{Encountered Problems}

\subsection{Limitations of \emph{\fullref{sec:requirement-multiple-administrators}}}\label{sec:limitations-of-requirement-multiple-administrators}

Requirement \ref{sec:requirement-multiple-administrators} states that multiple administrators should be able to access the application.

When authenticating users with SASL EXTERNAL, the client certificate extension field `xmppAddr' is interpreted as user \gls{jid} by the \gls{xmpp} server.

In practice, most \gls{xmpp-grid} \gls{broker} deployments will require an HTTP proxy in front of the \gls{xmpp} server as security measure\footnote{
More information on this can be found in Section~\fullref{sec:implemented-web-application-topology}.}.
Usually, the HTTP proxy can also be used to serve the \gls{broker} application.
Such an HTTP proxy might also accept multiple different client certificates.

If the client connects to the \gls{xmpp} server over secure WebSockets (WSS) in combination with SASL EXTERNAL, the WebSocket URL must already be authenticated, as most browsers do not permit certificate selection on background requests\footnote{\url{https://bugs.chromium.org/p/chromium/issues/detail?id=329884\#c24}}.
This might be achieved by serving the \gls{broker} from the same domain or by using client certificate policies\footnote{\url{https://support.google.com/chrome/a/answer/6080885?hl=en\#manage-certs}}.

As the proxy intercepts the TLS connection, it must verify the client certificate sent by the browser and establish a connection to the \gls{xmpp} server using a client certificate as well.
Therefore, the `xmppAddr' field of the proxy's client ceritifcate is used by the \gls{xmpp} server.
If multiple users should be differentiated on the \gls{xmpp} server, an HTTP proxy might choose different client certificates for connecting to the \gls{xmpp} server based on the web browser's client certificate `xmppAddr'.


\subsection{Limitations of \emph{\fullref{sec:requirement-audit-trail}}}

Actions of administrators should be traceable with an audit trail according to requirement \ref{sec:requirement-audit-trail}.

As outlined in Section~\ref{sec:limitations-of-requirement-multiple-administrators}, practical deployments of \gls{xmpp-grid} \glspl{broker} will mostly use a HTTP proxy.
Additionally, to handling the client authentication, the proxy can be used to keep an audit trail of client requests.
These requests can then be correlated with the query log on the \gls{xmpp} server.

Creating audit trails on the client side does not provide additional safety, as users might prevent trail entries by manipulating the client application.
Therefore, no such mechanism was implemented.


% - Logout (TLS)
% - Initial Topic Consumers and Providers + Initial Topic Consumers and Providers => 2 Step Process!
% - Openfire:
%   - Lost updates with OpenFire
%   - falsche Felder - speziell pubsub#node_type
%   - vgl. https://discourse.igniterealtime.org/t/wrong-field-type-of-pubsub-node-type-and-how-to-update-it/81596
% -> Fehlende Methoden/Funktionalität im Standard
%   - "Liste alle Topics" -> Geht nur hierarchisch
%   - Filtering von Persisted Items

\section{Code Quality}
As our \gls{xmpp-grid} \gls{broker} implementation is intended to be a maintainable, production-ready application rather than a prototype, we placed much emphasis on code quality.
The measures taken can broadly be divided into three categories: technical measures, strategic decisions and processes.

\subsubsection{Technical Measures and Strategic Decisions}
Using Angular and the default Angular CLI was mostly a strategic decision.
Deviating as less as possible from the standard configuration ensures long-term maintainability and relatively straight-forward upgrades to newer Angular versions.
Another benefit of the Angular CLI project setup is that it comes with ``codelyzer''\footnote{\url{http://codelyzer.com/}} (including ``tslint'') for static code analysis and style linting.

Apart from the built-in linting mechanism, we followed Angular's Style Guide~\cite{angular-style-guide}.
Using the JetBrains IDEs (IntelliJ Ultimate and Webstorm)\footnote{\url{https://www.jetbrains.com/}} turned out to be particularly helpful as they give quick feedback for frequent mistakes and even violations of the Angular Style Guide.

We would have loved to use more tools, especially for code metrics such as Lack of Cohesion of Methods (LCOM), and Afferent/Efferent Coupling.
Sadly, we were not able to find such tools that were actively maintained and work with TypeScript.

\subsubsection{Processes}

On the process side, we tried to work test driven as much as possible.
Doing so turned out to be harder than expected as Angular's component testing infrastructure and the actual calls are sometimes wide apart (see Section~\fullref{sec:testing}).

Another process we heavily relied on were code reviews.
Each change, for the documentation and code, was reviewed using GitHub pull-requests\footnote{\url{https://www.github.com/}}.
In most cases, minor changes were detected and addressed during these reviews.
Continuous integration with TravisCI\footnote{\url{https://travis-ci.com/}} ensured that these changes never contained compilation errors or failing tests.

We also regularly discussed architectural and structural questions in our retrospectives and standup meetings.

In general, writing clean, modular and testable code has been our main priority.

\section{Testing}\label{sec:testing}
% - Test
    % - Sinnvoll
    % - Abdeckung der Anforderungen
    % - Coverage
    % - Protokolle

% Viele Angular Component Tests -> Leider grosser gap zwischen test env und realität was testing etwas mühsam macht
% Integration/E2E-Tests mit Protractor
% - mühsam wegen BOSH da immer "polling"

\section{Documentation}
% - Installation
% - Benutzerdokumentation
% - Zielerreichung
% - Architectural Decisions