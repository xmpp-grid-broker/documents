% !TeX spellcheck = en_GB
\chapter{Implementation and Testing} % Realisierung und Test
\epigraph{Any fool can write code that a computer can understand. Good programmers write code that humans can understand.}{Martin Fowler}

\section{Problems}

% - Multiple Administrations (stanza.io)
% - Audit Trail (obsolete)
% - Logout (TLS)
% - Initial Topic Consumers and Providers + Initial Topic Consumers and Providers => 2 Step Process!
% - Openfire:
%   - Lost updates with OpenFire
%   - falsche Felder - speziell pubsub#node_type
%   - vgl. https://discourse.igniterealtime.org/t/wrong-field-type-of-pubsub-node-type-and-how-to-update-it/81596
% -> Fehlende Methoden/Funktionalität im Standard
%   - "Liste alle Topics" -> Geht nur hierarchisch
%   - Filtering von Persisted Items

\section{Code Quality}
% - Qualität des Codes
    % - Metriken
    % - Coding Guidelines
    % - Coding Reviews Richtlinien
    % - JavaDoc etc
    % - Patterns
    % - Code stimmt mit Arch. überein

% Typescript mit liter (tslint)
% Best-Practices von Angular befolgt dank Intellij Ultimate (`@angular/language-service`) und https://github.com/mgechev/codelyzer (ink. tslint)
% Modular und Testbar
% Doku mit Compodoc

\section{Complexity}
% - Komplexität und Umfang
    % - Metriken
\section{Testing}
% - Test
    % - Sinnvoll
    % - Abdeckung der Anforderungen
    % - Coverage
    % - Protokolle

% Viele Angular Component Tests -> Leider grosser gap zwischen test env und realität was testing etwas mühsam macht
% Integration/E2E-Tests mit Protractor
% - mühsam wegen BOSH da immer "polling"

\section{Documentation}
% - Installation
% - Benutzerdokumentation
% - Zielerreichung