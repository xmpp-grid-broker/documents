% !TeX spellcheck = en_GB
\chapter{Discussion and Conclusion}
\epigraph{Wisdom is not a product of schooling but of the lifelong attempt to acquire it.}{Albert Einstein}
\section{Achieved Result}

In this section, we describe the archived results during this thesis and how we managed to reach them.

\subsection{Implemented Requirements}

As listed in Table~\ref{tab:implemented-requirements}, we implemented about 86\% of the overall requirements that we had planned to accomplish. The five remaining requirements could not be implemented due to technical constraints as is discussed in depth in Section~\fullref{encountered-problems}. To compensate for it, we implemented two optional requirements.

\begin{table}[H]
    \begin{tabu}{X l}
        \toprule
        Requirement Group
        & implemented\\
        % & comment
        \midrule

        \fullref{sec:authentication}
        & \textit{partial (4/7)} \\
        %& Missing: "Multiple Administrators", "Audit Trail" and "Logout"\\

        \fullref{sec:list-topics}
        & \textit{partial (5/6)}\\
        % & without name filter and  optional features ("Limited Access")\\

        \fullref{sec:create-topic}
        & \textit{complete (1/1)}\\
        % & \\

        \fullref{sec:create-collection}
        & \textit{complete (3/3)}\\
        % & Without initial Consumers and Providers\\

        \fullref{sec:delete-topic}
        & \textit{complete (1/1)}\\
        % & \\

        \fullref{sec:delete-collection}
        & \textit{complete (3/3)}\\
        % & \\

        \fullref{sec:manage-subscriptions}
        & \textit{complete (5/5)}\\
        % & \\

        \fullref{sec:manage-affiliations}
        & \textit{complete (4/4)}\\
        % & \\

        \fullref{sec:manage-persisted-items}
        & \textit{partial (4/5)}\\
        % & Without filtering and "Delete Set of Persisted Item From a Topic"\\

        \fullref{sec:subscription-requests}
        & \textit{not implemented}\\
        % & \\

        \fullref{sec:validate-controller-config}
        & \textit{complete (2/2)}\\
        % & \\

        % \midrule
        \textbf{Total}
        & 32/37 $\approx 86\%$ \\
        % % & \\

    \end{tabu}
    \caption{Fulfilled requirements by groups}
    \label{tab:implemented-requirements}
\end{table}


\subsection{Architecture}

\subsubsection{Performance}
Due to our chosen architecture style (see Chapter~\fullref{sec:architecture}),
the main performance concern of \gls{xmpp-grid} administrators lies with the \gls{xmpp} server.

Our implementation submits queries to the \gls{xmpp} server in parallel whenever possible and reduces redundant queries via data sharing.
However, due to limitations in some \glspl{xep}, it was not possible to implement all requirements efficiently.
More details on these limitations can be found in Section~\fullref{encountered-problems}.

\subsubsection{Concurrency and Scalability}
Concurrency was not a primary concern as our application runs entirely in the user's web browser and \gls{xmpp} servers are responsible for managing concurrency in \gls{xmpp}.

As with concurrency, the scalability of an \gls{xmpp-grid} is the responsibility of the underlying \gls{xmpp} server, which was not part of this thesis.


\subsubsection{Usability}
Usability was an priority in our application and we implemented several features for ease of use.
A good example is the use of so-called bread-crumbs, which allow fast and direct navigation through different application levels.

We regret that it was not possible to conduct a usability test with a typical user during the thesis.

\subsubsection{Security}
In an expert review of our architecture, a good security level was attested.

To prevent risks due to misconfiguration or missing features of the \gls{xmpp} server or reverse proxy, we provide additional documentation alongside the application, containing recommendations for administrators. More details on this can be found in Section~\ref{sec:security-risk-mitigation} and `docs' folder in the source code repository.

\subsubsection{Architectural Decisions}
In this Section, we reflect on our \nameref{sec:architectural-decisions} and how they turned out.

\paragraph{Architecture Style}
Due to limitations of the \glspl{xep}, features like autocomplete and filtering could not be implemented.
This would probably have worked better with a server plug-in but would have resulted in close coupling to a specific XMPP server.

\paragraph{Platform}
The implementation of a web application proved as portable and flexible as intended.

\paragraph{SASL Authentication Strategy}
The use of SASL EXTERNAL proved to be suboptimal.
We discovered that due to policies in current web browsers and our chosen architecture, a reverse-proxy is nearly always required. The reason for this is that the \gls{xmpp} WebSocket or \gls{bosh} endpoint must already be authenticated.

In hindsight, to use SASL SCRAM with username and password would probably have eased the development and deployment of the application.

More details on this issue can be found in Section~\ref{sec:limitations-of-requirement-multiple-administrators}.

\paragraph{Role Management}
We are convinced that the decision to utilise collection nodes for role management is an ideal solution.
However, we were not able to verify this functionality, as Openfire has not implemented collection nodes according to the latest version of the \gls{publish-subscribe} \gls{xep} draft \cite{xep-0248}.

\paragraph{Web Application Communication Topology}
In general, using \gls{xmpp} directly from the web browser worked well.
However, due to the incomplete WebSocket implementation in Openfire and the browsers policies concerning SASL EXTERNAL, we had to use BOSH and an HTTP proxy in front of the \gls{xmpp} server.
See Section~\fullref{encountered-problems} for more details.

\paragraph{Frontend Framework}
The decision to use Angular with TypeScript in combination with the IntelliJ IDEA IDE has turned out to be an efficient and clean solution.

\paragraph{UI Library}
The decision for the spectre.css library provided us with a reasonable compromise regarding productivity and long-term maintainability.

\paragraph{Frontend Structure}
To split the application into multiple modules worked well and helped to strucutre the code.
We had to slightly modify the initial layering and design in the course of the project.

\paragraph{XMPP Client Library}
The Stanza.io \gls{xmpp} library has served its purpose.
We opened two pull requests with error corrections on Github \footnote{See \url{https://github.com/otalk/jxt-xmpp/pull/23} and \url{https://github.com/legastero/stanza.io/pull/264}}, which were quickly merged and released.

\subsection{Implementation}

\subsubsection{Tests}
As described in Section~\ref{sec:testing}, good tests and a solid test coverage are very important for a long-lived project.

To measure unit tests coverage, we used the istanbul coverage tool\footnote{\url{https://gotwarlost.github.io/istanbul/}}. We achieved a total of XX.X\% line coverage, thanks to our comprehensive set of unit tests. %TODO (also: line coverage is the least informative)

The code coverage achieved using the integration tests is not included in the test coverage, as no such tooling exists.

We wrote XX integration tests with a total of XXX lines of code. %TODO

In total, we have XX test code lines per line of application code.  %TODO ~> Rewrite when the exact numbers and judge the result


\section{Lessons Learned}
% - Documentation: Difficult to summarise standards;
% - Compodoc documentation tool was not a grand help - probably better suited for library/widget providers
% - We tried to systematically document arch. decisions, how did that turned out?
%   - Good, and helps to think about decisions; helps for structured planning
%   - We could have done more such decisions later in the project (structure in the modules; barrel imports etc.)
% + Docker Development environment and automation was very helpful in terms of production and portability for developers.
% - It paid off, writing many tests - but TDD sadly not possible with angular and stanza (api was not easy to mock - no TS types).

% - Project management
%   - Issue management good in general (JIRA)
%       - some issues with time estimates due to too large issues https://project.redbackup.org/browse/XGB-47 https://project.redbackup.org/browse/XGB-46
%       - We should have better specified tasks and created more and better todo lists
%       - difficult to estimate because we did not know a) Angular b) XMPP c) XMPP-Grids beforehand
%   + CI Setup / GitHub Reviews / Deployment project website


% -----
% - XMPP Standards with standard versions is BAD (e.g. split of pubsub nodes/collection)
% - XEPs are very interlinked and not all standards are actively extended (e.g. rely on deprecated standards)
% - XEPs are very much OPTIONAL or MAY - you can rely on nearly no features; many can not be queried feature discovery. XEPs have no comprehensive specified e.g. error cases.

% - XMPP-Grid standard uses different terminology than xmpp standards - somtimes overlaping terms. This makes it very difficult to comprehend it.


\section{Future work}
The result of our bachelor thesis is a working application, ready to prove itself in practice.
Even though all specified functionality was implemented, the user experience can still be further improved.

Conducting usability tests by observing administrators who manage \glspl{xmpp-grid} can reveal significant insights~\cite{krug:dont-make-me-think}.

To further improve the user experience, auto-complete for users and topics might be helpful.
As already discussed in Section~\fullref{encountered-problems}, this cannot be implemented efficiently due to shortcomings in the \gls{publish-subscribe} \gls{xep}.
One option would be to get involved and propose the required functionality in the \gls{xep} standardisation process.
A more short-term solution would be to tie the implementation closer to a specific \gls{xmpp} server that supports these features over proprietary APIs.
Alternatively, an unofficial \gls{xep} including corresponding server plugins can be specified and implemented.

Working around shortcomings of the \gls{xmpp} server implementations, lost updates for example (See Section~\fullref{sec:lost-updates}), could advance the usability as well.
However, It must be noted that adding more logic in the client contradicts the \gls{xmpp} philosophy that encourages simple client and complex server implementations \cite{definitive-guide-xmpp}.

\section{Conclusion}
\paragraph{The \gls{xmpp-grid} \gls{broker} application} enables administrators to configure \glspl{xmpp-grid} in a usable and productive way.
The modern web interface facilitates obtaining a comprehensive view of the configuration and structure of an \gls{xmpp-grid}.
Apart from improving the usability significantly, the application is also cross-platform and not tied to a particular \gls{xmpp} server implementation.

\paragraph{Our proposed architecture} has proven to work in practice.
Although the initial setup with a proxy server is not trivial, the architecture will pay off in practice regarding security and maintainability as reverse proxies are commonly used, and static sites are easy to maintain and upgrade.

\paragraph{Angular and stanza} turned out to be a good choice for the implementation.
Angular provides productive tools and a comprehensive testing infrastructure that allowed us to build an application that can be maintained efficiently in the long-term.

Stanza met most of our requirements concerning \gls{xmpp} support and allowed us to extend and improve it where needed.

Using Openfire as an \gls{xmpp} server backend was demanding at times due to the scanty implementation of the \gls{publish-subscribe} standard.
Some of these shortcomings, however, revealed problematic limitations of the standard which otherwise might not have been considered.

\paragraph{The Bachelor Thesis} went well from our point of view.
We were able to satisfy all critical requirements despite some limitations in the underlying standards and \gls{xmpp} server hindered our implementation.

\paragraph{In the future,} the application must prove itself in practice.
Based on feedback from users in the industry, the usability and feature set can further be refined.

To support some features efficiently, the implementation must either be bound to a specific \gls{xmpp} server or new extensions to the \gls{xmpp} standard must be proposed.

We hope that with the help of our implementation the IETF draft ``Using \gls{xmpp} for Security Information Exchange'' will become an established security standard used in practical industry applications.
