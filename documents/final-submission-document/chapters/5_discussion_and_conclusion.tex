% !TeX spellcheck = en_GB
\chapter{Discussion and Conclusion}
\epigraph{Wisdom is not a product of schooling but of the lifelong attempt to acquire it.}{Albert Einstein}
\section{Achieved Result}

% Requirements(?)
% - requirements: table with "top level requirements"

% Architecture
% - How did our architectural decisions turn out?: Positive and negative example
%    - SASL EXTERNAL mit Websockets / client certificate -> would be easier with username/pw -> In practice you need proxy, because popup; link to problems
%    - WebSockets with openfire did not work; webapi proxy would have some benefits, but "normal" web proxy are very stable.
%    - Architectue Style: Features like autocomplete and filtering could only be implemented with server plugin...
%    + Stanza is acceptable -> We do not use API but use direct messageing because of advanced features we require -> Our Pull Requests were merged fast
%    + Web Application with Angular: TS with IDE support has paid off.
%    + UI library was a help, but left us enough freedom; good decision regaring long life cycle.
%    ? Role Managment: right way, but no way to test
% - performance in terms of speed, scalability, concurrency, usability etc
% - good security attestet in reviews

% Implementation
% - tests: coverage etc. (see redbackup); how many lines tests/lines code ratio
% - Ausgewählte Implementierungsdetails/Metriken diskutieren (Bsp. Algorithmen, Datenstrukturen, Libraries, Architectural Hot Spots)

\section{Lessons Learned}
% - Documentation: Difficult to summarise standards;
% - Compodoc documentation tool was not a grand help - probably better suited for library/widget providers
% - We tried to systematically document arch. decisions, how did that turned out?
%   - Good, and helps to think about decisions; helps for structured planning
%   - We could have done more such decisions later in the project (structure in the modules; barrel imports etc.)
% + Docker Development environment and automation was very helpful in terms of production and portability for developers.
% - It paid off, writing many tests - but TDD sadly not possible with angular and stanza (api was not easy to mock - no TS types).

% - Project management
%   - Issue management good in general (JIRA)
%       - some issues with time estimates due to too large issues https://project.redbackup.org/browse/XGB-47 https://project.redbackup.org/browse/XGB-46
%       - We should have better specified tasks and created more and better todo lists
%       - difficult to estimate because we did not know a) Angular b) XMPP c) XMPP-Grids beforehand
%   + CI Setup / GitHub Reviews / Deployment project website


% -----
% - XMPP Standards with standard versions is BAD (e.g. split of pubsub nodes/collection)
% - XEPs are very interlinked and not all standards are actively extended (e.g. rely on deprecated standards)
% - XEPs are very much OPTIONAL or MAY - you can rely on nearly no features; many can not be queried feature discovery. XEPs have no comprehensive specified e.g. error cases.

% - XMPP-Grid standard uses different terminology than xmpp standards - somtimes overlaping terms. This makes it very difficult to comprehend it.


\section{Future work}
% - Would require further analysis with "real XMPP-Grids administrators"
% - Usability
%    - A next step would be usability tests
% - Some features require standard expansion and/or implementation in a xmpp server (see encountered problems)
% - Features that are in Backlog

\section{Conclusion}
\paragraph{The \gls{xmpp-grid} \gls{broker} application} enables administrators to configure \glspl{xmpp-grid}in a usable and productive way.
The modern web interface facilitates obtaining a comprehensive view of the configuration and structure of an XMPP-Grid.
Apart from improving the usability significantly, the application is also cross-platform and not tied to a particular XMPP server implementation.

\paragraph{Our proposed architecture} has proven to work in practice.
Although the initial setup with a proxy server is not trivial, the architecture will pay off in practice regarding security and maintainability as reverse proxies are commonly used, and static sites are easy to maintain and upgrade.

\paragraph{Angular and stanza} turned out to be a good choice for the implementation.
Angular provides productive tools and a working testing infrastructure that allowed us to build an application that can be maintained efficiently in the long-term.
Stanza met most of our requirements concerning XMPP and allowed us to extend and improve it where needed.
Using Openfire as an XMPP-Server backend was demanding at times due to the scanty implementation of the \gls{publish-subscribe} standard.
Some of these shortcomings, however, revealed problematic parts in the standards which otherwise might not have been considered.

\paragraph{The Bachelor Thesis} went well from our point of view.
We were able to satisfy all critical requirements despite some limitations in the underlying standards and XMPP-Server hindered our implementation.

\paragraph{In the future}, the application must prove itself in practice.
Based on feedback from users in the industry, the usability and feature set can further be refined.
The implementation must either be bound to a specific server implementation or new extensions to the XMPP standard must be proposed to implement some features efficiently