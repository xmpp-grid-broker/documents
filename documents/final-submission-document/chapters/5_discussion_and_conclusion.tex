% !TeX spellcheck = en_GB
\chapter{Discussion and Conclusion}
\epigraph{Wisdom is not a product of schooling but of the lifelong attempt to acquire it.}{Albert Einstein}
\section{Achieved Result}

% Requirements(?)
% - requirements: table with "top level requirements"

% Architecture
% - How did our architectural decisions turn out?: Positive and negative example
%    - SASL EXTERNAL mit Websockets / client certificate -> would be easier with username/pw -> In practice you need proxy, because popup; link to problems
%    - WebSockets with openfire did not work; webapi proxy would have some benefits, but "normal" web proxy are very stable.
%    - Architectue Style: Features like autocomplete and filtering could only be implemented with server plugin...
%    + Stanza is acceptable -> We do not use API but use direct messageing because of advanced features we require -> Our Pull Requests were merged fast
%    + Web Application with Angular: TS with IDE support has paid off.
%    + UI library was a help, but left us enough freedom; good decision regaring long life cycle.
%    ? Role Managment: right way, but no way to test
% - performance in terms of speed, scalability, concurrency, usability etc
% - good security attestet in reviews

% Implementation
% - tests: coverage etc. (see redbackup); how many lines tests/lines code ratio
% - Ausgewählte Implementierungsdetails/Metriken diskutieren (Bsp. Algorithmen, Datenstrukturen, Libraries, Architectural Hot Spots)

\section{Lessons Learned}

In this section, we describe unexpected project events and the lessons we learned from them.

\subsection{Project Course}

\subsubsection{Issues and Time Management}

In general, our issue management and time tracking with JIRA\footnote{\url{https://www.atlassian.com/software/jira}} and our Scrum-based approached worked very well.

While discussing time management issues in retrospective 3, we noted that we significantly underestimated the required time for several implementation issues.
Many implementation issues were also relatively large, in most cases estimated four to hours or more hours.

To address these estimation issues, we decided to create smaller issues and list tangible subtasks in the form of check-lists.
A check-list extension for JIRA facilitated this task.

Despite the smaller task sizes, estimating and specifying tasks precisely remained a challenge.
Our limited experience with the Angular framework and the XMPP ecosystem were undoubtedly large contributing factors.

\subsubsection{Documentation}

To accomplish high-quality documentation, we used GitHub pull requests to carry out peer reviews.
To simplify this process, we also set up continuous integration builds which always posted the latest stable documentation and appendices on the project website.
We think that this approach led to a high overall documentation standard.

It was difficult to summarise the technical background and describe our architecture due to the different terminology used by the XMPP and IETF standards.
We discuss this in Section~\ref{lessons-learned-standards}.

\subsection{Decisions}
\subsubsection{Architecture}

Architecture-relevant decisions were carried out and justified in the form of architectural design decisions \cite{architectural-design-decisions} (see Appendix~\ref{sec:architectural-decisions}).

This approach helped us to systematically document influences and plan the architecture in a structured way.

Retrospectively, we should have created more architectural decisions later on in the project, e.g. concerning barrel imports or to establish layering guidelines.

\subsection{Development, Frameworks and Tooling}

\paragraph{Test Driven Development} was not possible the way we had anticipated it.

Due to the use of Angular and stanza, the testing environment differed substantially from the actual application context.
Therefore, it was challenging to create tests before implementing most of the actual code structure.
A factor that also contributed to this difficulty was our lack of prior knowledge of the used frameworks.

Nevertheless, writing many tests proved to be very valuable during the project.
It helped us be confident during development and will be useful to future developers expanding the application.

\paragraph{The Docker Development Environment} has proven itself to be valuable for a quick start in development.
It provides developers with a very productive way to test modifications in a realistic yet portable environment.

\paragraph{Compodoc,} the tool we used to document and visualise the structure of our Angular application,
did not add as much additional value to the project documentation as we had hoped.

We assume that Compodoc is be better suited for Angular libraries than applications.

\subsection{Standards}\label{lessons-learned-standards}

During the course of this thesis, we learned several lessons about working with standards and in which way these standards posed challenges or supported our work.

\subsubsection{Terminology}

The \gls{xmpp-grid} standard uses \gls{sacm} terminology \cite{ietf-sacm-terminology-14}, whereas the \gls{xmpp} standard and all \glspl{xep} use a different terminology.
Most concepts and term definitions differ or overlap slightly, making it difficult to comprehend and connect both formats.
It also makes the use of a consistent terminology impossible, as some concepts from \gls{xmpp}/\glspl{xep} are not reflected in \gls{sacm} terminology and vice versa.

\subsubsection{\glspl{xep} Draft Versions}
Many of the used \gls{xmpp} Extension Procotols (\glspl{xep}) are not yet final but still in the draft phase.
Most notably, these include the Publish-Subscribe (XEP-0060 and XEP-0248), Result Set Management (XEP-0059) and BOSH (XEP-0206) \glspl{xep}.
Only the core \glspl{xep}, such as Service Discovery (XEP-0030) and Data Forms (XEP-0004), are declared as final.

Because many drafts have not received major updates (XEP-0059, for example, has not been modified for over 10 years) these drafts are treated as de facto standards in the community, neglecting the possibility of significant changes.
Sadly, not all drafts are stable.

A prominent example of a modified standard draft is the \gls{publish-subscribe} \gls{xep}.
In the last few years, significant changes have been made to and the concept of ``Collection Nodes'' was even extracted into a separate standard draft \cite{xep-0248}.
In our case, the Openfire \gls{xmpp} server implemented an older version of this \gls{xep}, not supporting all features that we planned to use in this thesis.

\subsubsection{Deprecated \glspl{xep}}

Many \glspl{xep} build on functionality specified by other, cross-referenced \glspl{xep}.
This is problematic, especially as some referenced standards are not active anymore.

An example for this is the PubSub Collection Nodes \gls{xep} \cite{xep-0248}, which currently has a \emph{deferred} status, but is still used in the latest version of the \gls{publish-subscribe} \gls{xep} \cite{xep-0060} which currently has a \emph{draft} status.

\subsubsection{Non-binding Standardisation}

Many features required for the implementation of an \gls{xmpp-grid} \gls{broker} are marked as optional in the corresponding \glspl{xep}.
To some extent, the availability of these features can be queried using the Feature Discovery mechanism \cite{xep-0030}, but not all optional features are exposed using Feature Discovery.

The \gls{publish-subscribe} \gls{xep} \cite{xep-0060} contains multiple such optional features.

Additionally, some features are not explicitly specified in the according \gls{xep}, but rather implicitly demonstrated using examples.

These limitations make it difficult, to rely on the availability of some features described in these \glspl{xep}


\section{Future work}
The result of our bachelor thesis is a working application, ready to prove itself in practice.
Even though all specified functionality was implemented, the user experience can still be further improved.

Conducting usability tests by observing administrators who manage \glspl{xmpp-grid} can reveal significant insights~\cite{krug:dont-make-me-think}.

To further improve the user experience, auto-complete for users and topics might be helpful.
As already discussed in Section~\fullref{encountered-problems}, this cannot be implemented efficiently due to shortcomings in the \gls{publish-subscribe} \gls{xep}.
One option would be to get involved and propose the required functionality in the \gls{xep} standardisation process.
A more short-term solution would be to tie the implementation closer to a specific \gls{xmpp} server that supports these features over proprietary APIs.
Alternatively, an unofficial \gls{xep} including corresponding server plugins can be specified and implemented.

Working around shortcomings of the \gls{xmpp} server implementations, lost updates for example (See Section~\fullref{sec:lost-updates}), could advance the usability as well.
However, It must be noted that adding more logic in the client contradicts the \gls{xmpp} philosophy that encourages simple client and complex server implementations \cite{definitive-guide-xmpp}.

\section{Conclusion}
\paragraph{The \gls{xmpp-grid} \gls{broker} application} enables administrators to configure \glspl{xmpp-grid} in a usable and productive way.
The modern web interface facilitates obtaining a comprehensive view of the configuration and structure of an \gls{xmpp-grid}.
Apart from improving the usability significantly, the application is also cross-platform and not tied to a particular \gls{xmpp} server implementation.

\paragraph{Our proposed architecture} has proven to work in practice.
Although the initial setup with a proxy server is not trivial, the architecture will pay off in practice regarding security and maintainability as reverse proxies are commonly used, and static sites are easy to maintain and upgrade.

\paragraph{Angular and stanza} turned out to be a good choice for the implementation.
Angular provides productive tools and a comprehensive testing infrastructure that allowed us to build an application that can be maintained efficiently in the long-term.

Stanza met most of our requirements concerning \gls{xmpp} support and allowed us to extend and improve it where needed.

Using Openfire as an \gls{xmpp} server backend was demanding at times due to the scanty implementation of the \gls{publish-subscribe} standard.
Some of these shortcomings, however, revealed problematic limitations of the standard which otherwise might not have been considered.

\paragraph{The Bachelor Thesis} went well from our point of view.
We were able to satisfy all critical requirements despite some limitations in the underlying standards and \gls{xmpp} server hindered our implementation.

\paragraph{In the future,} the application must prove itself in practice.
Based on feedback from users in the industry, the usability and feature set can further be refined.

To support some features efficiently, the implementation must either be bound to a specific \gls{xmpp} server or new extensions to the \gls{xmpp} standard must be proposed.

We hope that with the help of our implementation the IETF draft ``Using \gls{xmpp} for Security Information Exchange'' will become an established security standard used in practical industry applications.
