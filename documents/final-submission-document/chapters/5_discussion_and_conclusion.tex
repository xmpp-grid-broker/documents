% !TeX spellcheck = en_GB
\chapter{Discussion and Conclusion}
\epigraph{Wisdom is not a product of schooling but of the lifelong attempt to acquire it.}{Albert Einstein}
\section{Achieved Result}

In this section, we describe the achieved results during this thesis and how we managed to reach them.

\subsection{Implemented Requirements}

As listed in table~\ref{tab:implemented-requirements}, we implemented about 86\% of the overall requirements that we had planned to accomplish.
The five remaining requirements could not be implemented due to technical constraints as discussed in depth in section~\fullref{encountered-problems}.
To compensate for it, we implemented two optional requirements.

\begin{table}[H]
    \begin{tabu}{X l}
        \toprule
        Requirement Group
        & implemented\\
        % & comment
        \midrule

        \fullref{sec:authentication}
        & \textit{partial (4/7)} \\
        %& Missing: "Multiple Administrators", "Audit Trail" and "Logout"\\

        \fullref{sec:list-topics}
        & \textit{partial (5/6)}\\
        % & without name filter and  optional features ("Limited Access")\\

        \fullref{sec:create-topic}
        & \textit{complete (1/1)}\\
        % & \\

        \fullref{sec:create-collection}
        & \textit{complete (3/3)}\\
        % & Without initial Consumers and Providers\\

        \fullref{sec:delete-topic}
        & \textit{complete (1/1)}\\
        % & \\

        \fullref{sec:delete-collection}
        & \textit{complete (3/3)}\\
        % & \\

        \fullref{sec:manage-subscriptions}
        & \textit{complete (5/5)}\\
        % & \\

        \fullref{sec:manage-affiliations}
        & \textit{complete (4/4)}\\
        % & \\

        \fullref{sec:manage-persisted-items}
        & \textit{partial (4/5)}\\
        % & Without filtering and "Delete Set of Persisted Item From a Topic"\\

        \fullref{sec:subscription-requests}
        & \textit{not implemented}\\
        % & \\

        \fullref{sec:validate-controller-config}
        & \textit{complete (2/2)}\\
        % & \\

        % \midrule
        \textbf{Total}
        & 32/37 $\approx 86\%$ \\
        % % & \\

    \end{tabu}
    \caption{Fulfilled requirements by groups.}
    \label{tab:implemented-requirements}
\end{table}


\subsection{Architecture}

\subsubsection{Concurrency, Scalability and Performance}
Due to our chosen architecture style (see section~\fullref{sec:architecture}),
concurrency, scalability and performance are primarily the concern of the \gls{xmpp} server.

Our implementation submits queries to the \gls{xmpp} server in parallel whenever possible and reduces redundant queries via data sharing.

\subsubsection{Usability}
Usability was a priority in our application and we implemented several features for ease of use.
A good example is the use of so-called bread-crumbs, which allow fast and direct navigation through different application levels.

We regret that it was not possible to conduct a usability test with a typical user during the thesis.

\subsubsection{Security}
In an expert review of our architecture, a high level of security was attested.

To prevent risks due to misconfiguration or missing features of the \gls{xmpp} server or reverse proxy, we added additional documentation alongside the application, containing recommendations for administrators.
More details on this can be found in section~\fullref{sec:security-risk-mitigation} and the docs folder in the source code repository.

\subsubsection{Architectural Decisions}
In this section, we reflect on our \nameref{sec:architectural-decisions} and how they turned out.

\paragraph{Architecture Style}
Due to limitations of the \glspl{xep}, features like autocomplete and filtering could not be implemented.
This would probably have worked better with a server plug-in, but would have resulted in close coupling to a specific \gls{xmpp} server.

\paragraph{Platform}
The implementation of a web application proved portable and flexible as intended.

\paragraph{SASL Authentication Strategy}
The use of \gls{sasl-external} proved to be suboptimal.
We discovered that due to the chosen architecture and policies in current web browsers, a reverse-proxy is nearly always required (see section~\fullref{sec:limitations-of-requirement-multiple-administrators}).

In hindsight, to use \gls{sasl-scram} with username and password would probably have eased the development and deployment of the application.

\paragraph{Role Management}
We are convinced that the decision to model role management with collection nodes is an ideal solution.
However, we were not able to verify this functionality, as Openfire has not implemented collection nodes according to the latest version of the \gls{publish-subscribe} \gls{xep} draft \cite{xep-0248}.

\paragraph{Web Application Communication Topology}
In general, using \gls{xmpp} directly from the web browser worked well.
However, due to the incomplete WebSocket implementation in Openfire and the browsers policies concerning \gls{sasl-external}, we had to use \gls{bosh} and an \gls{http} proxy in front of the \gls{xmpp} server.
See section~\fullref{encountered-problems} for more details.

\paragraph{Frontend Framework}
The decision to use Angular with TypeScript in combination with the IntelliJ IDEA IDE has turned out to be an efficient and clean solution.

\paragraph{UI Library}
The decision for the Spectre.css\footnote{\url{https://picturepan2.github.io/spectre/}} library provided us with a reasonable compromise regarding productivity and long-term maintainability.

\paragraph{Frontend Structure}
To split the application into multiple modules worked well and helped to structure the code.
We had to slightly modify the initial design in the course of the project, to address the increasing complexity.

\paragraph{XMPP Client Library}
The Stanza.io \gls{xmpp} library has served its purpose.
We opened two pull requests with error corrections on GitHub\footnote{See \url{https://github.com/otalk/jxt-xmpp/pull/23} and \url{https://github.com/legastero/stanza.io/pull/264}}, which were quickly merged and released.

\subsection{Implementation}

\subsubsection{Tests}
As described in section~\fullref{sec:testing}, good tests and a solid test coverage are important for a long-lived project.

To measure unit tests coverage, we used the istanbul coverage tool\footnote{\url{https://gotwarlost.github.io/istanbul/}}. We achieved a total of 93.69\% statement coverage, thanks to our comprehensive set of unit tests.

The code coverage achieved using the integration tests is not included in the test coverage, as no such tooling exists.

In total, we have approximately 2.25 test code lines per line of application code.

\begin{table}[H]
    \begin{tabu}{X l}
        \toprule
        \textbf{Category}
        & \textbf{Lines of Code}\\
        \midrule

        Typescript Application Code
        & 2'060 \\

        HTML Application Code
        & 600 \\

        CSS Application Code
        & 163 \\
        \midrule
        \textbf{Total Application Code}
        & \textbf{2'823} \\
        \midrule

        Unit Test Code
        & 5'347 \\

        Integration Test Code
        & 994 \\

        \midrule
        \textbf{Total Test Code}
        & \textbf{6'341} \\
        \midrule

    \end{tabu}
    \caption{Lines of code by category excluding third-party code.}
    \label{tab:lines-of-code}
\end{table}

\begin{table}[H]
    \begin{tabu}{X l}
        \toprule
        \textbf{Test Category}
        & \textbf{Number of Tests}\\
        \midrule

        Unit Tests
        & 305 \\

        Integration Tests
        & 19 \\

        \midrule
        \textbf{Total}
        & \textbf{324} \\
        \midrule
    \end{tabu}
    \caption{Number of tests per test category.}
    \label{tab:no-of-tests}
\end{table}

\section{Lessons Learned}

In this section, we describe unexpected project events and the lessons we learned from them.

\subsection{Project Course}

\subsubsection{Issues and Time Management}

In general, our issue management and time tracking with JIRA\footnote{\url{https://www.atlassian.com/software/jira}} and our Scrum-based approach worked very well.

While discussing time management issues in retrospective 3, we noted that we significantly underestimated the required time for several implementation issues.
Many implementation issues were quite comprehensive, in some cases estimated at more than hours.

To address these estimation issues, we decided to create smaller issues and list tangible subtasks in the form of check-lists.
A check-list extension for JIRA facilitated this task.

Despite the reduced task sizes, estimating and specifying tasks precisely remained a challenge.
Our limited experience with the Angular framework and the \gls{xmpp} ecosystem were undoubtedly large contributing factors.

\subsubsection{Documentation}

To accomplish high-quality documentation, we used GitHub pull requests to carry out peer reviews.
To simplify this process, we also set up continuous integration builds which always posted the latest stable documentation and appendices on the project website.
We think that this approach led to a high overall documentation standard.

It was difficult to summarise the technical background and describe our architecture due to the different terminology used by the \gls{xmpp} and IETF standards.
We discuss this in section~\fullref{lessons-learned-standards}.

\subsection{Architectural Decisions}

Architecture-relevant decisions were carried out and justified in the form of architectural design decisions \cite{architectural-design-decisions} (see appendix~\fullref{sec:architectural-decisions}).

This approach helped us to systematically document influences and plan the architecture in a structured way.

Retrospectively, we should have made more architectural decisions later on in the project, e.g. concerning barrel imports or to establish layering guidelines.

\subsection{Development, Frameworks and Tooling}

\paragraph{Test Driven Development} was not possible in the way we had anticipated.

Due to the use of Angular, the testing environment differed substantially from the actual application context.
Therefore, it was challenging to create tests before implementing most of the actual code structure.
A factor that also contributed to this difficulty was our prior lack of Angular expertise.

Nevertheless, writing many tests proved to be very valuable.
It helped us to be confident during development and will be useful to future developers extending the application.

\paragraph{The Docker Development Environment} has proven to be valuable.
It provides developers with a very productive way to test modifications in a realistic yet portable environment.

\paragraph{Compodoc,} the tool we used to document and visualise the structure of our Angular application,
did not add as much value to the project documentation as we had hoped.

We assume that Compodoc is better suited for Angular libraries than applications.

\subsection{Standards}\label{lessons-learned-standards}

During the course of this thesis, we learned valuable lessons about working with standards and about the way these standards pose challenges or support development.

\subsubsection{Terminology}

The \gls{xmpp-grid-standard} uses \gls{sacm} terminology \cite{ietf-sacm-terminology-14}, whereas the \gls{xmpp} standard and all \glspl{xep} use a different terminology.
Most concepts and term definitions differ or overlap slightly, making it difficult to comprehend and connect both formats.
It also makes the use of a consistent terminology impossible, as some concepts from \gls{xmpp}/\glspl{xep} are not reflected in \gls{sacm} terminology and vice versa.

\subsubsection{\glspl{xep} Draft Versions}
Many of the used \gls{xmpp} Extension Procotols (\glspl{xep}) are not yet final but still in the draft phase.
Most notably, these include the Publish-Subscribe (XEP-0060 and XEP-0248), Result Set Management (XEP-0059) and \gls{bosh} (XEP-0206) \glspl{xep}.
Only the core \glspl{xep}, such as Service Discovery (XEP-0030) and Data Forms (XEP-0004), are declared final.

Because many drafts have not received major updates (XEP-0059, for example, has not been modified for over 10 years) these drafts are treated as de facto standards in the community, neglecting the possibility of significant changes.
Unfortunately, not all drafts are stable.

A prominent example of a modified standard draft is the \gls{publish-subscribe} \gls{xep}.
In the last few years, significant changes have been made and the concept of ``Collection Nodes'' was even extracted into a separate standard draft \cite{xep-0248}.
In our case, the Openfire \gls{xmpp} server implemented an older version of this \gls{xep}, not supporting all features that we planned to use in this thesis.

\subsubsection{Deprecated \glspl{xep}}

Many \glspl{xep} build on functionality specified by other, cross-referenced \glspl{xep}.
This is problematic, especially as some referenced standards are not active anymore.

An example is the PubSub Collection Nodes \gls{xep} \cite{xep-0248}, which currently has a \emph{deferred} status, but is still used in the latest version of the \gls{publish-subscribe} \gls{xep} \cite{xep-0060}, which currently has a \emph{draft} status.

\subsubsection{Non-Binding Standardisation}

Many features required for the \gls{xmpp-grid-broker} implementation are marked as optional in the corresponding \glspl{xep}.
To some extent, the availability of these features can be queried using the feature discovery mechanism \cite{xep-0030}, but not all optional features are exposed in this way.

The \gls{publish-subscribe} \gls{xep} contains multiple such optional features.

Additionally, some features are not explicitly specified in the according \gls{xep}, but rather implicitly demonstrated using examples.

These limitations make it difficult to rely on the availability of some features described in these \glspl{xep}.


\section{Future work}
The result of our bachelor thesis is a fully functional application, ready to prove itself in production.
Even though all specified functionality was implemented, the user experience can still be further improved.

Conducting usability tests by observing administrators who manage \glspl{xmpp-grid} can reveal significant insights~\cite{krug:dont-make-me-think}.

To further improve the user experience, auto-complete for users and topics might be helpful.
As already discussed in section~\fullref{encountered-problems}, this cannot be implemented efficiently due to shortcomings in the \gls{publish-subscribe} \gls{xep}.
One option would be to propose the required functionality in the \gls{xep} standardisation process.
A more short-term solution would be to tie the implementation closer to a specific \gls{xmpp} server that supports these features over proprietary APIs.
Alternatively, an unofficial \gls{xep} including corresponding server plugins can be specified and implemented.

Working around shortcomings of the \gls{xmpp} server implementations, lost updates for example (see section~\fullref{sec:lost-updates}), could advance the usability as well.
However, it must be noted that adding more logic in the client contradicts the \gls{xmpp} philosophy that encourages simple clients and complex server implementations~\cite{definitive-guide-xmpp}.

\section{Conclusion}
\paragraph{The XMPP-Grid broker application} enables administrators to configure \glspl{xmpp-grid} in a straight-forward and productive way.
The modern web interface facilitates obtaining a comprehensive view of the configuration and structure of an \gls{xmpp-grid}.
Apart from improving the usability significantly, the application is also cross-platform and not tied to a particular \gls{xmpp} server implementation.

\paragraph{Our proposed architecture} has proven to work in practice.
Although the initial setup with a proxy server is complex, the architecture will pay off in practice regarding security and maintainability as reverse proxies are commonly used, and static sites are easy to maintain and upgrade.

\paragraph{Angular and Stanza.io} turned out to be a good choice for the implementation.
Angular provides productive tools and a comprehensive testing infrastructure that allowed us to build an application that can be maintained efficiently in the long-term.

Stanza.io met most of our requirements concerning \gls{xmpp} support and allowed us to extend and improve it where needed.

Using Openfire as an \gls{xmpp} server backend was demanding at times due to the scanty implementation of the \gls{publish-subscribe} standard.
Some of these shortcomings, however, revealed problematic limitations of the standard which otherwise might not have been considered.

\paragraph{The Bachelor Thesis} went well from our point of view.
Not only were we able to reach all major requirements, but also deliver a robust and ready-to-use solution.

\paragraph{In the future,} the application must prove itself in practice.
Based on feedback from users in industry, the usability and feature set can further be refined.

To support some features efficiently, the implementation must either be bound to a specific \gls{xmpp} server or new extensions to the \gls{xmpp} standard must be proposed.

We hope that with the help of our implementation the IETF draft ``Using \gls{xmpp} for Security Information Exchange'' will become an established security standard used in practical industry applications.
\label{lastpage} %TODO: This label should be positioned below above the last paragraph