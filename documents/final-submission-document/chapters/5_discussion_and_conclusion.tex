% !TeX spellcheck = en_GB
\chapter{Discussion and Conclusion}
\epigraph{Wisdom is not a product of schooling but of the lifelong attempt to acquire it.}{Albert Einstein}
\section{Achieved Result}

% Requirements(?)
% - requirements: table with "top level requirements"

% Architecture
% - How did our architectural decisions turn out?: Positive and negative example
%    - SASL EXTERNAL mit Websockets / client certificate -> would be easier with username/pw -> In practice you need proxy, because popup; link to problems
%    - WebSockets with openfire did not work; webapi proxy would have some benefits, but "normal" web proxy are very stable.
%    - Architectue Style: Features like autocomplete and filtering could only be implemented with server plugin...
%    + Stanza is acceptable -> We do not use API but use direct messageing because of advanced features we require -> Our Pull Requests were merged fast
%    + Web Application with Angular: TS with IDE support has paid off.
%    + UI library was a help, but left us enough freedom; good decision regaring long life cycle.
%    ? Role Managment: right way, but no way to test
% - performance in terms of speed, scalability, concurrency, usability etc
% - good security attestet in reviews

% Implementation
% - tests: coverage etc. (see redbackup); how many lines tests/lines code ratio
% - Ausgewählte Implementierungsdetails/Metriken diskutieren (Bsp. Algorithmen, Datenstrukturen, Libraries, Architectural Hot Spots)

\section{Lessons Learned}
% - Documentation: Difficult to summarise standards;
% - Compodoc documentation tool was not a grand help - probably better suited for library/widget providers
% - We tried to systematically document arch. decisions, how did that turned out?
%   - Good, and helps to think about decisions; helps for structured planning
%   - We could have done more such decisions later in the project (structure in the modules; barrel imports etc.)
% + Docker Development environment and automation was very helpful in terms of production and portability for developers.
% - It paid off, writing many tests - but TDD sadly not possible with angular and stanza (api was not easy to mock - no TS types).

% - Project management
%   - Issue management good in general (JIRA)
%       - some issues with time estimates due to too large issues https://project.redbackup.org/browse/XGB-47 https://project.redbackup.org/browse/XGB-46
%       - We should have better specified tasks and created more and better todo lists
%       - difficult to estimate because we did not know a) Angular b) XMPP c) XMPP-Grids beforehand
%   + CI Setup / GitHub Reviews / Deployment project website


% -----
% - XMPP Standards with standard versions is BAD (e.g. split of pubsub nodes/collection)
% - XEPs are very interlinked and not all standards are actively extended (e.g. rely on deprecated standards)
% - XEPs are very much OPTIONAL or MAY - you can rely on nearly no features; many can not be queried feature discovery. XEPs have no comprehensive specified e.g. error cases.

% - XMPP-Grid standard uses different terminology than xmpp standards - somtimes overlaping terms. This makes it very difficult to comprehend it.


\section{Future work}
The result of our bachelor thesis is a working application, ready to prove itself in practice.
Even though all specified functionality was implemented, the user experience can still be further improved.

Conducting usability tests by observing administrators who manage \glspl{xmpp-grid} can reveal significant insights~\cite{krug:dont-make-me-think}.

To further improve the user experience, auto-complete for users and topics might be helpful.
As already discussed in Section~\fullref{encountered-problems}, this cannot be implemented efficiently due to shortcomings in the \gls{publish-subscribe} \gls{xep}.
One option would be to get involved and propose the required functionality in the \gls{xep} standardisation process.
A more short-term solution would be to tie the implementation closer to a specific \gls{xmpp} server that supports these features over proprietary APIs.
Alternatively, an unofficial \gls{xep} including corresponding server plugins can be specified and implemented.

Working around shortcomings of the \gls{xmpp} server implementations, lost updates for example (See Section~\fullref{sec:lost-updates}), could advance the usability as well.
However, It must be noted that adding more logic in the client contradicts the \gls{xmpp} philosophy that encourages simple client and complex server implementations \cite{definitive-guide-xmpp}.

\section{Conclusion}
% Similar to Redbackup

% - Verbesserungen für die  Zielgruppe im Kontext
%    - Usability
%    - Eases configuration of XMPP-Grids a lot
% - Goal was reached
% - Some limits due to standards etc. (reference lesseons learned section)
% - Main reason and use case for XMPP PubSub is missing in standards - this would have helped us to set priorities and build a sensible architecture. XMPP has a lot statefullness.
