% !TeX spellcheck = en_GB
\newcommand{\code}{\texttt}
\chapter{Introduction}
\label{sec:introduction}

\epigraph{Every accomplishment starts with the decision to try.}{unknown}
In this chapter, present the motivation and legitimisation of our thesis and highlight the delimitations.

\section{Motivation}
In this section, we legitimate this thesis and explain the value and applicability of our proposed solution.

\subsection{Present Situation}
The IETF standard draft \emph{Using \gls{xmpp} for Security Information Exchange} \cite{ietf-mile-xmpp-grid-05} as summarised in section \ref{sec:ietf-internet-draft-using-xmpp-for-security-information-exchange} defines a protocol to exchange security-relevant information between endpoints.
The draft was created by the Managed Incident Lightweight Exchange (MILE) Working Group to support computer and network security incident management.

To demonstrate the viability of the draft a rapid prototype was developed in November~2017~\cite{xmpp-grid-prototype}.

\subsection{Problem and Vision}
Currently, there exists no implementation of the \gls{xmpp} grid draft management functionality which is ready for production use regarding usability and security.

To solve this problem, a graphical interface with bindings to a suitable \gls{broker} must be proposed and implemented.
The interface should permit network administrators to manage and review \glspl{topic}, persisted items and \glspl{platform}.
Additionally, subscription and publishing permissions of \glspl{topic} and \glspl{platform} must be manageable.

\section{Delimitation} % Abgrenzung
As described in the Task Description, the focus of this thesis is on the evaluation, design and implementation of the XMPP Grid Broker.
Adding missing functionality or fixing complex bugs in existing server or client implementations are out of scope.
