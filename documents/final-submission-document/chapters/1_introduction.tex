% !TeX spellcheck = en_GB
\newcommand{\code}{\texttt}
\chapter{Introduction}
\label{sec:introduction}

\epigraph{Every accomplishment starts with the decision to try.}{unknown}
In this chapter, we present the motivation and legitimisation of our thesis and highlight the scope delimitations.

\section{Motivation}
In this first section, we shall legitimate this thesis and explain the value and applicability of our proposed solution.

\subsection{Present Situation}
The Internet Engineering Task Force (IETF) standard draft \emph{Using \gls{xmpp} for Security Information Exchange} \cite{ietf-mile-xmpp-grid-05} as summarised in Section~\ref{sec:ietf-standard-draft-using-xmpp-for-security-information-exchange} defines a protocol to exchange security-relevant information between endpoints.
The draft was created by the Managed Incident Lightweight Exchange (MILE) Working Group to support computer and network security incident management.

Hereafter, we refer to this IETF standard draft as \emph{\gls{xmpp}-Grid standard}.

To demonstrate the viability of the \gls{xmpp-grid} standard a rapid prototype was developed in November~2017~\cite{xmpp-grid-prototype}.

\subsection{Problem and Vision}
Currently, no implementation of the \gls{xmpp-grid} standard management functionality exists which is ready for production use regarding usability and security.

To solve this problem, a graphical interface with bindings to a suitable \gls{broker} must be proposed and implemented.
The interface should permit network administrators to manage and review \glspl{topic}, persisted items and \glspl{platform}.
Additionally, subscription and publishing permissions of \glspl{topic} and \glspl{platform} must be manageable.

The graphical interface supports administrators to better understand the underlying hierarchy and affiliations of \glspl{topic}, as well as their security implications.
As the interface uses familiar terminology known to an administrator, no in-depth understanding of the underlying \gls{xmpp} technology is required to configure and comprehend an \gls{xmpp-grid}.
Finally, the interface also provides better usability than existing command line interfaces and \gls{xmpp} configuration software, which leads to fewer configuration mistakes and improved efficiency.

We hope that with the help of our implementation the IETF draft ``Using \gls{xmpp} for Security Information Exchange'' will become an established security standard used in practical industry applications.

\section{Scope Delimitation} % Abgrenzung
As described in the \nameref{sec:task-description}, the focus of this thesis is on the evaluation, design and implementation of the \gls{xmpp-grid-broker}.
Adding missing functionality or fixing complex bugs in existing server or client implementations are beyond the scope of this thesis.
