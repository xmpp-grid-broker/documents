% !TeX spellcheck = en_GB
\chapter{Analysis}
\epigraph{Without requirements or design, programming is the art\\of adding bugs to an empty text file.}{Louis Srygley}
\section{Terminology}
Taking into account that developers and operators of security reporting systems are the intended audience for this thesis,
we mostly use the  Security Automation and Continuous Monitoring (SACM) terminology~\cite{ietf-sacm-terminology-14}
and thereby follow the same guidelines as the \gls{xmpp-grid} standard~\cite{ietf-mile-xmpp-grid-05}.

\section{Technical Background}\label{sec:technical-background}

The following sections introduce the underlying \gls{xmpp} protocol and the relevant extensions (\glspl{xep}) used by the \gls{xmpp-grid} standard as well as a summary of it and the corresponding \gls{xmpp} terminology.

\subsection{\gls{xmpp} (eXtensible Messaging and Presence Protocol)}
The Extensible Messaging and Presence Protocol (in short \gls{xmpp}) is an open protocol that enables the near-real-time exchange of small data between any network endpoints, hereafter called \glspl{platform}~\cite{rfc6120}.
While originally designed as an Instant Messaging (IM) protocol, \gls{xmpp} can be used for a wide range of data exchange applications~\cite{ieee-xplore-stream-xml-xmpp}.

\gls{xmpp} is made of small building blocks defined in the core protocol~\cite{rfc6120} and numerous extensions called \glspl{xep}~\cite{xep-0001}.
The core is comprised of functionality for setup and encryption of communication channels, \gls{xml} streams, error handling and more. Additional functionality such as \gls{service-discovery}~\cite{xep-0030} and \gls{publish-subscribe}~\cite{xep-0060} are defined in separate extensions.

Although \gls{xmpp} supports peer-to-peer communication, it is often used in a traditional client-server architecture.
A client (\gls{platform}) can send data to any addressable entity (any other \glspl{platform}) using \Gls{jabber} Identifiers, hereafter called \gls{jid}. If the receiving \gls{jid} has a different domain from the current server (\gls{controller}), the message is forwarded to the responsible \gls{xmpp} server under its domain~\cite{rfc6120}.

The data exchanged over \gls{xmpp} is \gls{xml}, which makes the protocol structured and extensible, but leads to some protocol overhead.
\gls{xmpp} communicates over unidirectional data streams with a server, which are basically long-lived \gls{tcp} connections.
The client opens a channel to the server over this connection, and the server opens one back (i.e. \code{<stream>} XML tags). In both streams, an XML document is opened after the connection is established.
During the conversation, an arbitrary amount of \glspl{stanza} (specified XML child elements) are written to the stream.
Before a connection may be terminated, the root element is closed (i.e. \code{</stream>}) and both streams form valid XML documents~\cite{rfc6120}\cite{professional-xmpp}.

The core \gls{stanza} types are \glspl{message}~(\code{<message/>}), \gls{presence}~(\code{<presence/>}) and\\
\gls{info-query}~(\code{<iq/>}).
\Glspl{message} can contain arbitrary data similar to email but are optimised for immediate delivery.
\Gls{presence} \glspl{stanza} deal with network availability and the propagation of user presence information.
An \gls{info-query} \gls{stanza} consists of a request and response (similar to the GET and POST HTTP methods), which is used for feature negotiation, configuration and general information exchange.
Because of these coarse semantics, \gls{xmpp} provides a generalized communication layer~\cite{rfc6120}\cite{ieee-xplore-stream-xml-xmpp}.

Figure~\ref{fig:xmpp-overview} illustrates an example setup with two servers and three clients.

\begin{figure}[h]
	\centering
	\includegraphics[width=0.8\linewidth]{resources/xmpp_overview.pdf}
	\caption{Two \gls{xmpp} domains (servers), one with two users and one with a single mobile user.}
	\label{fig:xmpp-overview}
\end{figure}

\subsection{Relevant \gls{xmpp} Extensions}

The \gls{xmpp-grid} standard~\cite{ietf-mile-xmpp-grid-05} is based on multiple \glspl{xep}, most notably \gls{publish-subscribe}. In this section, we give an overview of the most relevant used \glspl{xep}.

\paragraph{XEP-0004: \gls{data-forms}} is a flexible protocol that can be used in workflows such as service configuration as well as for application-specific data description and reporting. The protocol provides form processing, common field types and extensibility mechanisms~\cite{xep-0004}.

\paragraph{XEP-0030: \gls{service-discovery}} enables entities to discover information about the identity and capabilities of other entities, e.g. whether the entity is a server or not, or items associated with an entity, e.g. a list of \gls{publish-subscribe} nodes~\cite{xep-0030}.

\paragraph{XEP-0059: \gls{result-set-management}} allows entities to manage the receipt of large result sets, e.g. by paging through the result or limiting the number of results. \gls{result-set-management} is often desired when dealing with large dynamic result sets, as from service discovery or publish-subscribe, and when time or other resources are limited~\cite{xep-0059}.

\subsubsection{XEP-0060: \gls{publish-subscribe}}
The \gls{publish-subscribe} Extension, hereafter referred to as \gls{pubsub} or \gls{broker}, enables \gls{xmpp} entities (\gls{provider}) to broadcast information via \glspl{topic} to subscribed entities (\gls{consumer})~\cite{xep-0060}.

Nodes, hereafter referred to as \glspl{topic}, are the communication hubs. Entities can create \glspl{topic} and configure them, e.g. set up subscription timeouts or limit publishing and subscription rights. The configuration mechanism is based on data forms (XEP\babelhyphen{nobreak}0004). An \gls{xmpp}-Server \emph{may} restrict node creation to certain entities, which means that possibly not every \gls{xmpp}-Server that supports \gls{publish-subscribe} also implements this feature \cite{rfc2119}.

The protocol defines a hierarchy of six affiliations of which only the implementation of `owner' and `none' is \emph{required}.
Implementing the remaining four affiliations is \emph{recommended}.
An owner of a \gls{topic} can manage the subscriptions and affiliations of other entities associated with a given \gls{topic}.

To simplify the creation of \glspl{topic}, \gls{pubsub} defines five \gls{topic} access models (`node access models') that \emph{should} be available: `open', `presence', `roaster', `authorize' and `whitelist'.

The open model allows uncontrolled access while `presence' and `roaster' are specific for IM. Using the authorize model, the owner has to approve all subscription requests. The whitelist model enables the owner to maintain a list of entities that are allowed to subscribe.

\section{Requirements Analysis}
% NFR, priorisierung, Testbarkeit, Accessibility
% Anforderungen abgenommen?

We collected the functional requirements in the form of user stories.
User stories are an established and widespread concept for describing and managing requirements in agile software projects.
In comparison to traditional tools for requirement analysis, user stories are more concise, leaving more space for changes. \cite{wirdemann2017scrum}

We created some separate user stories for non-functional requirements.
Additional non-functional requirements can be added during the project in the form of constraints. \cite{wirdemann2017scrum}

In the early phase of the project, we collected an initial set of user stories based on the task description and discussions with Prof.\ Dr.\ Steffen.
This initial set covered the creation and deletion of \glspl{topic} as well as  granting publish and subscribe privileges.
All user stories are listed in Appendix~\fullref{sec:requirements}.

After setting broad priorities, Prof.\ Dr.\ Steffen approved the initial set of user stories that then served as the basis for the architectural concept draft.

Detailed implementation prioritisation of the user stories are carried out before every sprint in agreement with Prof.\ Dr.\ Steffen.

\section{Domain Analysis}

\subsection{IETF Standard Draft: Using \gls{xmpp} for Security Information Exchange}\label{sec:ietf-standard-draft-using-xmpp-for-security-information-exchange}
This IETF standard draft describes how the \gls{xmpp} protocol can be used for the exchange and distribution of security-relevant information between network devices with the XEPs discussed in Section~\ref{sec:technical-background}. The standard presumes use of the Incident Object Description Exchange Format (IODEF)~\cite{rfc7970}.

One of the primary motivation for using \gls{xmpp} for this task is the fast propagation of such security-relevant data.
Using \gls{xmpp} for such a task also comes with its downsides. Most notably, because the \gls{xmpp} server (\gls{broker}/\gls{controller}) is the central configuration component in charge of managing access permission, its compromisation has serious consequences.

The standard describes a trust model, a thread model as well as specific countermeasures, e.g. to use at least \gls{tls} 1.2. These countermeasures also define restrictions of the \gls{xmpp} protocol and its extensions, e.g. by limiting the \gls{topic} access models of \gls{pubsub} to whitelist and authorized only~\cite{ietf-mile-xmpp-grid-05}.

\subsubsection{Information Exchange in the IODEF}

The \gls{xmpp-grid} standard states that `although [the exchanged] information can take the form of any structured data (XML, JSON, etc.), this document illustrates the principles of \gls{xmpp}-Grid with examples that use the Incident Object Description Exchange Format (IODEF)'~\cite{ietf-mile-xmpp-grid-05}.

As IODEF is not strictly defined nor explicitly recommended by the \gls{xmpp-grid} standard, the implementation of this protocol is not within the scope of this thesis.

\subsection{Domain Specific Language}

Figures~\ref{fig:architecturedslgriddraft} and \ref{fig:architecturedslxeps} present an overview of the relevant interactions and relationships between the different components as specified in the \gls{xmpp-grid} standard~\cite{ietf-mile-xmpp-grid-05} and as used in the referenced XEPs (see Section~\ref{sec:technical-background}).

\begin{figure}[h]
\centering
\includegraphics[width=\linewidth]{resources/architecture_dsl_grid_draft}
\caption[DSL of \gls{xmpp-grid} standard]{Domain Specific Language of the \gls{xmpp-grid} standard}
\label{fig:architecturedslgriddraft}
\end{figure}

\begin{figure}[h]
\centering
\includegraphics[width=\linewidth]{resources/architecture_dsl_xeps}
\caption[DSL of used \gls{xmpp} XEPs]{Domain Specific Language of used \gls{xmpp} XEPs}
\label{fig:architecturedslxeps}
\end{figure}
